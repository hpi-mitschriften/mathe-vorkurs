\documentclass[12pt,a4paper]{article}
\usepackage{mathtools}
\usepackage{amsmath}
\usepackage{amsfonts}
\usepackage{amssymb}
\usepackage[makeroom]{cancel}
\usepackage{tabularx}
\usepackage{bookmark}
\usepackage{adjustbox}
\usepackage{pgf, tikz}
\usetikzlibrary{shapes}
\usepackage[german]{babel}
\usepackage[]{hyperref}
\hypersetup{
    pdftitle = {Grundbegriffe und Schreibweisen},
    pdfauthor = {Yoan Tchorenev},
    bookmarksnumbered = true,
    bookmarksopen = true,
    bookmarksopenlevel = 3,
    colorlinks=false
}
\setlength{\parindent}{0pt}
\title{Grundbegriffe und Schreibweisen}
\usepackage[left=2cm, right=2cm, top = 2cm, bottom=2cm]{geometry}
\author{Yoan Tchorenev}

\newcommand{\rom}[1]{\uppercase\expandafter{\romannumeral #1 \relax}}
\begin{document}
\begin{titlepage}
	\maketitle
	\tableofcontents
\end{titlepage}

\section{Logik}
\subsection{Begriffe}
\begin{description}
	\item[Aussage:] Eine Aussage ist eine Formel oder ein sprachliches Gebilde dem genau ein Wahrheitswert zugeordnet werden kann.
	
	\item[Warheitswerte] Genau der Eine oder der Andere \\
		\begin{tabular}{c|c}
			\textbf{F}alsch & \textbf{W}ahr \\
			0 & 1 \\
			$\bot$ & $\top$ \\
			\textbf{L}ow & \textbf{H}igh
		\end{tabular}
	
	\item[Aussagevariable] A,B,C etc. stehen für eine Aussage
	
	\item[Junktoren] (Verknüpfer)
		\begin{description}
			\item[Negation] $\neg A$ \, ''nicht'', ''NOT'', auch: $A$, $\bar{A}$, $A'$ \\
				\begin{tabularx}{\linewidth}{c|X}
					\begin{tabular}[t]{c|c}
						A & $\neg A$  \\ \hline
						0 & 1 \\
						1 & 0
					\end{tabular} &
					Mathematisch: $\neg A = (A + 1) \bmod 2$ \\ \hline
				\end{tabularx}
			
			\item[Konjunktion] $A\wedge B$ \, ''A und B'', ''AND'', auch $A\cdot B$, AB \\
				\begin{tabularx}{\linewidth}{c|X}
					\begin{tabular}[t]{c|c||c}
						A & B & $A \wedge B$ \\ \hline\hline
						0 & 0 & 0 \\ \hline
						0 & 1 & 0 \\ \hline
						1 & 0 & 0 \\ \hline
						1 & 1 & 1
					\end{tabular} &
					\begin{tabular}[t]{ll}
						Mathematisch: & $A \wedge B = A \cdot B$ \\
						Kommutativ: & $A \wedge B \equiv B \wedge A$ \\
						Assoziativ: & $A \wedge (B \wedge C) \equiv (A \wedge B) \wedge C$ \\
						Idempotent: & $A \wedge A \equiv A$ \\
						$A \wedge \bot \equiv \bot$ & $A \wedge \top \equiv A$
					\end{tabular} \\ \hline
				\end{tabularx}	
			
			\item[Disjunktion] $A\vee B$ \, ''A oder B'' (inklusiv), ''OR'' \\
				\begin{tabularx}{\linewidth}{c|X}
					\begin{tabular}[t]{c|c||c}
						A & B & $A \vee B$ \\ \hline\hline
						0 & 0 & 0 \\ \hline
						0 & 1 & 1 \\ \hline
						1 & 0 & 1 \\ \hline
					1 & 1 & 1
					\end{tabular} &
					\begin{tabular}[t]{ll}
						Mathematisch: & $A \vee B = \min(A+B;1)$ \\
						Kommutativ: & $A \vee B \equiv B \vee A$ \\
						Assoziativ: & $A \vee (B \vee C) \equiv (A \vee B) \vee C$ \\
						Idempotent: & $A \vee A \equiv A$ \\
						$A \vee \bot \equiv A $ & $A \vee \top \equiv \top$
					\end{tabular} \\ \hline
				\end{tabularx}
				
			\item[Kontravalenz] $A\dot{\vee}B$ \, ''entweder A, oder B'' (exclusiv), ''XOR'', auch: $A\oplus B$ \\
				\begin{tabularx}{\linewidth}{c|X}
					\begin{tabular}[t]{c|c||c}
						A & B & $A \dot{\vee} B$ \\ \hline\hline
						0 & 0 & 0 \\ \hline
						0 & 1 & 1 \\ \hline
						1 & 0 & 1 \\ \hline
						1 & 1 & 0
					\end{tabular} &
					\begin{tabular}[t]{ll}
						Mathematisch: & $A \dot{\vee} B = (A + B)\bmod 2$ \\
						Kommutativ: & $A \dot{\vee} B \equiv B \dot{\vee} A$ \\
						Assoziativ: & $A \dot{\vee} (B \dot{\vee} C) \equiv (A \dot{\vee} B) \dot{\vee} C$ \\
						$\neg$ Idempotent: & $A \dot{\vee} A \equiv \bot$ \\
						$A \dot{\vee} \bot \equiv A$ & $A \dot{\vee} \top \equiv \neg A$
					\end{tabular} \\ \hline
				\end{tabularx}
				
			\item[Konditional] $A\Rightarrow B$ ''wenn A dann B'' auch ''Subjunktion'', ''Implikation'', ''IMPLY'' \\
				\begin{tabularx}{\linewidth}{c|c|X}
					\begin{tabular}[t]{c|c||c}
						A & B & $A \Rightarrow B$ \\ \hline\hline
						0 & 0 & 1 \\ \hline
						0 & 1 & 1 \\ \hline
						1 & 0 & 0 \\ \hline
						1 & 1 & 1
					\end{tabular} &
					\begin{tabular}[t]{c|c}
						A & B \\ \hline
						Prämisse & Konklusion \\
						Voraussetzung & Konsequenz \\
						hinreichende & notwendige
					\end{tabular} &
					$A \Rightarrow B \equiv \neg A \vee B$ \newline
					Mathematisch: $A \Rightarrow B = \newline \min((A + 1) \bmod 2 + B; 1)$ \\ \hline
				\end{tabularx} \\
				\begin{tabular}[t]{rl}
					Eigenschaften & $A \Rightarrow \bot \equiv \neg A; \quad A \Rightarrow \top \equiv \top; \quad \bot \Rightarrow A \equiv \top; \quad \top \Rightarrow A \equiv A$ \\
					Kontraposition & $A \Rightarrow B \equiv \neg B \Rightarrow \neg A$ \\
					Abtrennungsregel & $(A \wedge (A \Rightarrow B)) \Rightarrow B$ \\
					Kettenschluss & $((A \Rightarrow B) \wedge (B \Rightarrow C)) \Rightarrow (A \Rightarrow C)$
				\end{tabular}							
				
			\item[Bikonditional] $A \Leftrightarrow B$ ''A genau dann, wenn B'', ''XNOR'', auch ''Aquivalenz'' $\equiv$ \\
				\begin{tabularx}{\linewidth}{c|X}
					\begin{tabular}[t]{c | c || c}
						A & B & $A \Leftrightarrow B$ \\ \hline\hline
						0 & 0 & 1 \\ \hline
						0 & 1 & 0 \\ \hline
						1 & 0 & 0 \\ \hline
						1 & 1 & 1
					\end{tabular} &
					\begin{tabular}[t]{ll}
						Mathematisch: & $A \Leftrightarrow B = (A + B + 1)\bmod 2$ \\
						Kommutativ: & $A \Leftrightarrow B \equiv B \Leftrightarrow A$ \\
						Assoziativ: & $A \Leftrightarrow (B \Leftrightarrow C) \equiv (A \Leftrightarrow B) \Leftrightarrow C$ \\
						$\neg$ Idempotent: & $A \Leftrightarrow A \equiv \top$ \\
						$A \Leftrightarrow \bot \equiv \neg A$ & $A \Leftrightarrow \top \equiv A$
					\end{tabular} \\ \hline
				\end{tabularx}
		\end{description}	
\end{description}

\subsection{Terme}
	\begin{description}
		\item[Tautologie] Ein Term W heißt heißt Tautologie, wenn er nur den Wahrheitswert 1 hat.
		\item[Äquivalenz] Zwei aussagenlogische Terme W und V heißen logisch äquivalent $$W \equiv V$$ wenn sie gleichen Wahrheitswert haben. Zwei Terme W und V sind genau dann logisch äquivalent, wenn der Term $W\Leftrightarrow V$ Tautologie ist.
	\item[Klammern] Regeln:
			\begin{itemize}
				\item Außenklammern können weggelassen werden
				\item Die stärke der Zeichen ist konventionell: $\neg > \wedge > \vee$. D.h.:
					$$\neg A \vee B \wedge C \equiv (\neg A) \vee (B \wedge C)$$
				\item $\wedge$ und $\vee$ sind distributiv zueinander:
					$$A \wedge (A \vee B) \equiv (A \wedge B) \vee (A \wedge C)$$
					$$A \vee (A \wedge B) \equiv (A \vee B) \wedge (A \vee C)$$
				\item $\wedge$ ist distributiv über $\dot{\vee}$:
					$$A \wedge (B \dot{\vee} C) \equiv (A \wedge B) \dot{\vee} (A \wedge C)$$
			\end{itemize}
		\item[De-Morganische Gesetze] $$\overline{A \wedge B} \equiv \overline{A} \vee \overline{B}$$ $$\overline{A \vee B} \equiv \overline{A} \wedge \overline{B}$$
	\end{description}

\subsection{Beweise}
	\begin{description}
		\item[Aussageform] Haben die Form einer Aussage, enthalte aber Variablen.
		$$3 + x = 5;\ A(x);\ B(x;y)$$
		\begin{itemize}
			\item werden zu Aussagen, wenn die Variablen belegt werden. Für die Variablen ist ein eingrenzender Grundbereich vorzugeben. Z.B.: $x \in \mathbb{N}$
			\item Wie Aussagen kann man Aussageformen miteinander Verknüpfen (mit Junktoren) und man erhält neue Aussageformen
		\end{itemize}
		
		\item[Quantoren] Außer der Belegung der Variablen mit Werten gibt es noch andere Möglichkeiten aus einer Aussageform eine Aussage zu machen. Ein Grundbereich $M$ muss vorgegeben sein.
		
		''Für alle x aus $M$ gilt A(x)'' \newline
		Für alle $x \in \mathbb{N}$ gilt $3 + x = 5$ (falsche Aussage) kurz mit Allquantor $\forall$ :
		$$(\forall x \in \mathbb{N})\ 3 + x = 5$$
		
		''Es existiert ein x aus $M$ mit A(x)'' \newline 
		Es existiert (mindestens) ein $x \in \mathbb{N}$ mit $3 + x = 5$ (wahre Aussage) kurz mit Existenzquantor $\exists$ :
		$$(\exists x \in M)\ 3 + x = 5$$
		
		''Es existiert höchsten ein x aus $M$ mit A(x)''
		$$(\forall x)(\forall y)\ (A(x) \wedge A(y) \Rightarrow x = y)$$
		
		''Es existiert genau ein x aus $M$ mit A(x)''
		$$(\exists ! x) A(x) \equiv ((\exists x) A(x)) \wedge ((\forall x)(\forall y)\ (A(x) \wedge A(y) \Rightarrow x = y))$$
	\end{description}

\section{Mengenlehre}
\subsection{Begriffe}
Georg Cantor (1845-1918)
	\begin{description}
		\item[Cantors naive Mengendefinition] Unter einer Menge verstehen wir eine Zusammenfassung von wohldefinierten Objekten $m$ unserer Anschauung oder unseres Denkens welche die Elemente von $M$ genannt werden, zu einem einheitlichen Ganzen.
		\item[Schreibweise]\ 
			\begin{itemize}
				\item $m \in M$ ($m$ ist Element von $M$)
				\item $m \not\in M$ ($m$ ist nicht Element von $M$, $\neg\ m \in M$)
			\end{itemize}
		\item[Mengendarstellung] verschiedene Möglichkeiten:
			\begin{itemize}
				\item allgemein mitels Eigenschaft $E(m)$ (Aussageform) $A=\lbrace m|E(m) \rbrace$ bzw.
					$$A = \lbrace m \in M | E(m) \rbrace = \lbrace m | m \in M \wedge E(m) \rbrace$$
				\item explizit für Menge mit wenigen endlich vielen Elementen:
					 $$A=\lbrace a, b, c\rbrace$$
			\end{itemize}
		\item[Problem] Man darf nicht alle möglichen Zusammenfassungen bilden. Z.B.: die Menge aller Mengen die sich nicht selbst enthalten:
			$$R=\lbrace M | M \not \in M \rbrace$$
			$$R \in R \Leftrightarrow R \not \in R \equiv \bot$$
		\item[Lösung] Axiomatischer Aufbau der Mengenlehre
			\begin{description}
				\item[Extensionalitätsaxiom] Zwei Mengen A und B sind genau dann gleich, wenn sie die selben Elemente haben:
					$$A = B \Leftrightarrow (\forall x)(x \in A \Leftrightarrow x \in B)$$
				\item[Leere Menge] $\emptyset = \lbrace x | x \not = x\rbrace = \lbrace\rbrace$
				\item[Einermenge] $A=\lbrace a \rbrace$, $A = \lbrace x | x = a \rbrace$, $A \not = a$
				\item[Zweiermenge] $A=\lbrace a; b \rbrace$, $A = \lbrace x|(x=a \vee x=b) \wedge a \not = b \rbrace$
				\item[andere Mengen] \ 
					\begin{itemize}
						\item $\mathbb{N} = \lbrace 0;1;2;3;\dots \rbrace$ natürliche Zahlen
						\item $\mathbb{Z} = \lbrace \dots; (-1);0;1;\dots \rbrace$ ganze Zahlen
						\item $\mathbb{Q}$ rationale Zahlen
						\item $\mathbb{R}$ reelle Zahlen
						\item $\mathbb{C}$ komplexe Zahlen
					\end{itemize}
			\end{description}
		\item[Betrag] Anzahl der Elemente in der Menge (bei endlichen Mengen)
		\item[Teilmenge] $A \subseteq B \Leftrightarrow (\forall x)(x \in A \Rightarrow x \in B)$
			$$A \subseteq B \wedge B \subseteq C \Rightarrow A \subseteq C$$
			$$A \subseteq B \wedge B \subseteq A \Rightarrow A = B$$
		\item[Echte Teilmenge] $A \subset B$ oder $A \subsetneqq B \Leftrightarrow (\forall x)(x \in A \Rightarrow x \in B) \wedge A \not = B $
		\item[disjunkt] Die Mengen $A$ und $B$ heißen disjunkt (elementfremd) wenn: $A \cap B = \emptyset$
		\item[Kardinalität] Mächtigkeit
			\begin{description}
				\item[gleichmächtig] Zwei Mengen $A;B$ heißen gleich mächtig, wenn es eine bijektive Funktion $f : A \longrightarrow B$ gibt.
					$$A \sim B \Leftrightarrow (\exists f : A \longrightarrow B)$$
					$$A \sim B \wedge B \sim C \Rightarrow A \sim C$$
				\item[endlich] Menge $A$ heißt endlich, wenn $|A| \in \mathbb{N}$
				\item[abzählbar unendlich ] Eine Menge $A$ heißt abzählbar unendlich, wenn $$\mathbb{N} \sim A \wedge \exists f : \mathbb{N} \longrightarrow A\ \textrm{(bijektiv)}$$
				\item[nicht abzählbar unendlich] Meine Menge heißt nicht abzählbar unendlich, wenn sie weder endlich noch abzählbar unendlich ist.
				\item[Potenzmengen]$M \not \sim \mathcal{P}(M)$\\ Beweis:
					Angenommen es gäbe eine bijektive Funktion $f : A \longrightarrow \mathcal{P}(M)$ und
						$$A = \lbrace x \in M | x \not \in f(x) \rbrace \subset M$$
						Wir nehmen an dass $(\exists x \in M)\ f(x) = A$
						\begin{itemize}
							\item wenn $x \in f(x) $ dann $x \not \in A$ wegen $x \not \in f(x)$. Widerspruch da: $x \not \in A = x \not \in f(x)$
							\item wenn $x \not \in f(x)$ dann $x \in A$ wegen $x \in M$. Widerspruch da: $x \not \in A = x \not \in f(x)$ 
						\end{itemize}
			\end{description}			 
	\end{description}
	
\subsection{Operationen auf Mengen}
	\begin{description}
		\item[Vereinigung] $A \cup B = \lbrace x | x \in A \vee x \in B \rbrace$ \\
			\begin{tabular}{l|l|l}
				\adjustbox{valign = t}{
				\begin{tikzpicture}[thick, set/.style = {circle, minimum size = 2cm, fill=red}]
					\node [set, label={90:$A$}] (A) at (-0.5,0) {};
					\node [set, label={90:$B$}] (B) at (0.5,0) {};
					\draw (-0.5,0) circle(1);
					\draw (0.5,0) circle(1);
				\end{tikzpicture}
				} &
				\adjustbox{valign = t}{
				\begin{tikzpicture}[thick, set/.style = {circle, minimum size = 2cm, draw = black, fill=red}]
					\node [set, label={90:$A$}] (A) at (-1.1,0) {};
					\node [set, label={90:$B$}] (B) at (1.1,0) {};
				\end{tikzpicture}} &	
				$\begin{array}{r c l}
					|A \cup B| & = & |A| + |B \setminus A| \\
              		 & = & |B| + |A \setminus B|
				\end{array}$
			\end{tabular}
		\item[Durchschnitt] $A \cap B := \lbrace x | x \in A \wedge x \in B \rbrace$ \\
			\begin{tabular}{l|l|l}
				\adjustbox{valign = t}{
				\begin{tikzpicture}[thick, set/.style = {circle, minimum size = 2cm, draw = black}]
					\begin{scope}
						\clip (-0.5,0) circle(1);
						\fill[red] (0.5, 0) circle (1);
					\end{scope}
					\node [set, label={90:$A$}] (A) at (-0.5,0) {};
					\node [set, label={90:$B$}] (B) at (0.5,0) {};
				\end{tikzpicture}
				} &
				\adjustbox{valign = t}{
				\begin{tikzpicture}[thick, set/.style = {circle, minimum size = 2cm, draw = black}]
					\node [set, label={90:$A$}] (A) at (-1.1,0) {};
					\node [set, label={90:$B$}] (B) at (1.1,0) {};
				\end{tikzpicture}
				} &
				$\begin{array}{r c l}
					|A \cap B| & = & |A| - |A \setminus B| \\
              		 & = & |B| - |B \setminus A|
				\end{array}$
			\end{tabular}
		\item[Mengendifferenz] $A \setminus B = \lbrace x | x \in A \wedge x \not \in B \rbrace$ \\
			\begin{tabular}{l|l|l}
				\adjustbox{valign = t}{
				\begin{tikzpicture}[thick, set/.style = {circle, minimum size = 2cm, draw = black}]
					\begin{scope} [even odd rule]
						\clip (-0.5,0) circle(1) (0.5,0) circle(1);1
						\fill [red] (-0.5,0) circle (1);
					\end{scope}
					\node [set, label={90:$A$}] (A) at (-0.5,0) {};
					\node [set, label={90:$B$}] (B) at (0.5,0) {};
				\end{tikzpicture}
				} &
				\adjustbox{valign = t}{
				\begin{tikzpicture}[baseline=(current bounding box.north), thick, set/.style = {circle, minimum size = 2cm, draw = black}]
					\node [set, fill = red, label={90:$A$}] (A) at (-1.1,0) {};
					\node [set, label={90:$B$}] (B) at (1.1,0) {};
				\end{tikzpicture}
				} &
			\end{tabular}
		\item[symmetrische Differenz] $A \Delta B = (A \setminus B) \cup (B \setminus A)$ \\
			\begin{tabular}{l|l|l}
				\adjustbox{valign = t}{
				\begin{tikzpicture}[thick, set/.style = {circle, minimum size = 2cm, draw = black}]
					\fill [even odd rule, red] (-0.5,0) circle (1) (0.5,0) circle (1);
					\node [set, label={90:$A$}] (A) at (-0.5,0) {};
					\node [set, label={90:$B$}] (B) at (0.5,0) {};
				\end{tikzpicture}
				} &
				\adjustbox{valign = t}{
				\begin{tikzpicture}[baseline=(current bounding box.north), thick, set/.style = {circle, minimum size = 2cm, draw = black, fill = red}]
					\node [set, label={90:$A$}] (A) at (-1.1,0) {};
					\node [set, label={90:$B$}] (B) at (1.1,0) {};
				\end{tikzpicture}
				} & 
				$|A \Delta B| = |A \setminus B| + |B \setminus A|$
			\end{tabular}
			
		
		\item[Potenzmengen] $\mathcal{P}(A) \coloneqq \lbrace B|B \subseteq A \rbrace;\ |\mathcal{P}(A)| = 2^{|A|}$
		
		\item[ungeordnets Paar] $\lbrace a,b \rbrace = \lbrace c,d \rbrace \Rightarrow (a=c \wedge b=d) \vee (a=d \wedge b=c)$
		
		\item[geordnetes Paar] $\lbrace a,b \rbrace = \lbrace c,d \rbrace \Rightarrow a=c \wedge b=d$ (Das geht!)
			
		\item[Mengenprodukt] $A \times B = \lbrace (a,b)|a \in A \wedge b \in B \rbrace$ (nicht Kommutativ, (strenggenommen) nicht assoziativ)
			\begin{align*}
				(A \times B) \times C &\not = A \times (B \times C) \\
				((a,b),c) &\not = (a,(b,c))
			\end{align*}
			Gegeben sein
			$$A = \lbrace 1, 2 \rbrace$$
			$$B = \lbrace a, b, c \rbrace$$
			dann ist:
			$$A \times B = \lbrace (1,a),(2,a)(1,b)(2,b),(1,c)(2,c) \rbrace$$ \\
			$|A \times B| = |A| \cdot |B|$
	\end{description}

\section{Funktionen}
Funktionen sind im wesentlich Zuordnungen.

\subsection{Begriffe}
	\begin{description}
		\item[Definition] Zur Definition einer Funktion $f$ braucht man drei Dinge
			\begin{itemize}
				\item Menge $A$, der Definitionsbereich von $f$, $A = D_f$
				\item Menge $B$, der Wertevorrat von $f$, $B = W_f$
				\item Eine Zuordnung, die jedem $a \in A$ genau ein Element $b \in B$ zuordnet \\
					Schreibweise: $b = f(a)$ bzw. $a \longmapsto f(a)$ \\
					Mathematisch wird diese Zuordnung gegeben durch eine Menge von geordneten Paaren
					$$\textrm{Graph}(f) = \lbrace(a,f(a)) | a \in A \rbrace \subseteq A \times B$$
					mit den Eigenschaften:
					\begin{itemize}
						\item $(\forall a \in A)(\exists b \in B)\ (a;b) \in \textrm{Graph}(f)$ (Vollständigkeit)
						\item $(\forall a \in A)(\forall b_1,b_2 \in B)\ (a;b_1);(a;b_2) \in \textrm{Graph}(f)\Rightarrow b_1 = b_2$ (Eindeutigkeit)
					\end{itemize}
			\end{itemize}
		\item[Schreibweise]
			\begin{alignat*}{3}
				f :\ &A \longrightarrow &&B \quad , \quad a &&\longmapsto f(a) = \cdots \\
				&D_f &&W_v && \textrm{Graph}
			\end{alignat*}
		\item[Bild] Die Menge aller Funktionswerte von $f$. $\lbrace f(a) | a \in A \rbrace = \lbrace b \in B  | (\exists a \in A) b = f(a) \rbrace \subseteq B$ 
		\item[surjektiv] $(\forall b \in B)(\exists a \in A) \ f(a) = b$ \\
			\begin{tabularx}{\linewidth}{l|X}
				\adjustbox{valign = t}{
				\begin{tikzpicture}[thick, set/.style = {ellipse, minimum width = 2cm, minimum height = 4cm, draw = black, align = center}, element/.style = {circle, draw = black, minimum size = 0.7, outer sep = 0.05cm}]
					\node [set, label={90:$A$}] (A) at (-1.5,0) {};
					\node [set, label={90:$B$}] (B) at (1.5,0) {};
					\node [element] (1) at (-1.5, 1.5) {1};
					\node [element] (2) at (-1.5, 0.5) {2};
					\node [element] (3) at (-1.5, -0.5) {3};
					\node [element] (4) at (-1.5, -1.5) {4};
					\node [element] (A) at (1.5, 1.5) {A};
					\node [element] (B) at (1.5, 0.5) {B};
					\node [element] (C) at (1.5, -0.5) {C};
					\draw [->] (1) to (A);
					\draw [->] (2) to (B);
					\draw [->] (3) to (C);
					\draw [->] (4) to (C);
				\end{tikzpicture}
				} &
				Für jedes Element in $B$ existiert (mindestens) ein Urbild in $A$. Für jede rein surjektive Abbildung gilt:
					$$|A|>|B|$$ \\ \hline
			\end{tabularx}
		\item[injektiv] $(\forall a_1,a_2 \in A) (a_1 \not = a_2 \Rightarrow f(a_1) \not = f(a_2))$ \\
			\begin{tabularx}{\linewidth}{l|X}
				\adjustbox{valign = t}{
				\begin{tikzpicture}[thick, set/.style = {ellipse, minimum width = 2cm, minimum height = 4cm, draw = black, align = center}, element/.style = {circle, draw = black, minimum size = 0.7, outer sep = 0.05cm}]
					\node [set, label={90:$A$}] (A) at (-1.5,0) {};
					\node [set, label={90:$B$}] (B) at (1.5,0) {};
					\node [element] (1) at (-1.5, 1.5) {1};
					\node [element] (2) at (-1.5, 0.5) {2};
					\node [element] (3) at (-1.5, -0.5) {3};
					\node [element] (A) at (1.5, 1.5) {A};
					\node [element] (B) at (1.5, 0.5) {B};
					\node [element] (C) at (1.5, -0.5) {C};
					\node [element] (D) at (1.5, -1.5) {D};
					\draw [->] (1) to (A);
					\draw [->] (2) to (B);
					\draw [->] (3) to (D);
				\end{tikzpicture}
				} &
				Für jede zwei Elemente in $A$ gilt, dass wenn sie verschieden von einander sind, dann auch ihre Funktionswerte von $f$ verschieden sind. Also hat jedes Element in $B$ höchstens ein Urbild. Für jede rein injektive Abbildung gilt:
					$$|A|<|B|$$ \\ \hline
			\end{tabularx} 
		\item[bijektiv]  surjektiv $\wedge$ injektiv: $(\forall b \in B)(\exists ! a \in A)\ f(a) = b$ \\
		\begin{tabularx}{\linewidth}{l|X}
				\adjustbox{valign = t}{
				\begin{tikzpicture}[thick, set/.style = {ellipse, minimum width = 2cm, minimum height = 4cm, draw = black, align = center}, element/.style = {circle, draw = black, minimum size = 0.7, outer sep = 0.05cm}]
					\node [set, label={90:$A$}] (A) at (-1.5,0) {};
					\node [set, label={90:$B$}] (B) at (1.5,0) {};
					\node [element] (1) at (-1.5, 1.5) {1};
					\node [element] (2) at (-1.5, 0.5) {2};
					\node [element] (3) at (-1.5, -0.5) {3};
					\node [element] (4) at (-1.5, -1.5) {4};
					\node [element] (A) at (1.5, 1.5) {A};
					\node [element] (B) at (1.5, 0.5) {B};
					\node [element] (C) at (1.5, -0.5) {C};
					\node [element] (D) at (1.5, -1.5) {D};
					\draw [->] (1) to (A);
					\draw [->] (2) to (B);
					\draw [->] (3) to (C);
					\draw [->] (4) to (D);
				\end{tikzpicture}
				} &
				Für jedes Element in $B$ existiert genau ein Urbild in $A$. Für jede bijektive Abbildung gilt:
					$$|A|=|B|$$ \\ \hline
			\end{tabularx}
		\item[Identitätsfunktion] $id_A : A \longrightarrow A , a \longmapsto a$ z.B. $f(x) = x$
		\item[Komposition] $f : A \longrightarrow B;\ g : B \longrightarrow C$
			$$(g \circ f) : A \longrightarrow C, a \longmapsto g(f(a))$$
			\begin{alignat*}{3}
				f : A \longrightarrow B \Rightarrow &f &&= f \circ id_A &&= id_A \circ f \\
				&f(a) &&= f(id_A(a)) &&= id_A(f(a))
			\end{alignat*}		
	\end{description}

\subsection{Umkehrfunktion}
	\begin{description}
		\item[Umkehrbarkeit] (im engeren sinne) $f : A \longrightarrow B$
			$$\leftrightarrow (\exists g : B \longrightarrow A) g \circ f = id_A \wedge f \cdot g = id_B$$
			$$(\forall a \in A)\ g(f(a)) = a$$
			$$(\forall b \in B)\ f(g(b)) = b$$
			Die Funktion $g : B \longrightarrow A$ heißt dann Umkehrfunktion von f, geschrieben $g = f^{-1}$.
			$$f^{-1} \not = (f)^{-1}$$
			Satz: Eine Funktion $f : A \longrightarrow B$ ist genau dann umkehrbar (i.e.s), wenn sie bijektiv ist.
		\item[Umkehrbarkeit in der Analysis] Eine Funktion $f : A \longrightarrow B$ heißt Umkehrbar, wenn die zugehörige Funktion $f : A \longrightarrow \textrm{Bild}(f)$ umkehrbar ist.
			Satz: Eine Funktion $f : A \longrightarrow B$ ist genau dann umkehrbar (i.w.s), wenn sie injektiv ist.
		\item[Quadratische] $f : \mathbb{R} \longrightarrow \mathbb{R}, x \longmapsto x^2$ \\
			\begin{tabularx}{\linewidth}{l X}
				\adjustbox{valign = t}{
				\begin{tikzpicture}
					\draw[->] (-3, 0) -- (3, 0) node[right] {$x$};
					\draw[->] (0, -1) -- (0, 3) node[above] {$y$};
					\draw[domain=-1.7:1.7, smooth, variable=\x, red] plot (\x, \x * \x);
					\draw[domain=-1:3, smooth, variable=\x, gray, dotted] plot (\x, \x);
					\draw[domain=0:3, smooth, variable=\x, blue]  plot(\x, {\x^(0.5)});
				\end{tikzpicture}
				} &
				$f^* : \mathbb{R}^+_0\longrightarrow \mathbb{R}^+_0$ (bijektiv) \newline
				$f^{*-1} : \mathbb{R}_0^+ \longrightarrow \mathbb{R}_0^+, x \longmapsto \sqrt{x}$
			\end{tabularx}
		\item[Exponentialfunktion] $\exp_B : \mathbb{R} \longrightarrow \mathbb{R}^+, x \longmapsto B^x$ mit Basis $B > 1$\\
			\begin{tabularx}{\linewidth}{l X}
				\adjustbox{valign = t}{
				\begin{tikzpicture}[baseline]
					\draw[->] (-3, 0) -- (3, 0) node[right] {$x$};
					\draw[->] (0, -3) -- (0, 3) node[above] {$y$};
					\draw[domain=-3:1.1, smooth, variable=\x, red] plot ({\x}, {exp(\x)});
					\draw[domain=-3:3, smooth, variable=\x, gray, dotted] plot ({\x}, {\x});
					\draw[domain=0.05:3, smooth, variable=\x, blue]  plot(\x,{ln((\x))});
				\end{tikzpicture}
				} &
				$\exp_B^{-1}:\mathbb{R}_0^+ \longrightarrow \mathbb{R}, x \longmapsto \log_B(x)$
			\end{tabularx}	
	\end{description}
	
\section{Zahlen}

\subsection{Sprachunterschiede}
	\begin{tabular}{|l|l|l|} \hline
		& deutsch & US-Englisch \\ \hline \hline
		$10^6$ & Million & million \\ \hline
		$10^9$ & Milliarde & billion \\ \hline
		$10^{12}$ & Billion & trillion \\ \hline
		$10^{15}$ & Billiarde & quadrillion \\ \hline
		$10^{18}$ & Trillion & quintillion \\ \hline
	\end{tabular}
\subsection{natürliche Zahlen}
	$\mathbb{N} = \lbrace 0, 1, 2, 3, \dots \rbrace$
	\begin{description}
		\item[unendlichkeits Axiom] Es gibt unendliche Mengen
		\item[Peano-Axiome] 5 Stück:
			\begin{itemize}
				\item $0 \in \mathbb{N}$, null ist eine natürliche Zahl
				\item es gibt eine Nachfolgerfunktion $s : \mathbb{N} \longrightarrow \mathbb{N}$
				\item $s$ ist injektiv
				\item $0 \not \in \textrm{Bild}(s)$, Null ist nicht Nachfolger einer natürlichen Zahl
				\item Für jede Menge $M \subseteq \mathbb{N}$ gilt:
				$$(0 \in \mathbb{N} \wedge (\forall n \in \mathbb{N})(n \in M \Rightarrow s(n) \in M)) \Rightarrow M = \mathbb{N}$$
				Modifikation: steht $M \subseteq \mathbb{N}$ kann man das auch als Eigenschaft $E_M(n)$ ausdrücken.
				$$E_M(n) \Leftrightarrow n \in M$$
			\end{itemize}
		\item[Vollständige Induktion] am Beispiel für einen Beweis der Gaußschen Summenformel
			\begin{description}
				\item[Induktionsvoraussetzung] Die Annahme: $A(n) \Leftrightarrow 1 + 2 + \dots + n = \frac{n(n+1)}{2}$
				\item[Induktionsanfang] Der Beweis, dass der Anfang gültig ist: $A(1) = 1$
				\item[Induktionsbehauptung] Das Einsetzen von $(n + 1)$ für $n$:
					$$A(n + 1) \Leftrightarrow 1 + \dots + n + (n + 1)= \frac{(n + 1)((n + 1)+1)}{2}$$
				\item[Induktionsschritt] Zeigen, dass aus der Induktionsvoraussetzung
					$$A(n) \Leftrightarrow 1 + \dots + n = \frac{n(n+1)}{2}$$
					die Induktionsbehauptung
					$$A(n + 1) \Leftrightarrow 1 + \dots + n + (n + 1) = \frac{(n + 1)((n + 1)+1)}{2}$$
					folgt. In diesem speziellen Fall:
					\begin{align*}
						A(n + 1) \Leftrightarrow 1 + \dots + n + (n + 1) &= \frac{n(n + 1)}{2} + (n + 1) \\
						&= \frac{n(n + 1) + 2(n + 1)}{2} \\
						&= \frac{(n + 1)(n + 2)}{2} \\
						&= \frac{(n + 1)((n + 1)+1)}{2}
					\end{align*}
			\end{description}
		\item[Addition] $m \in \mathbb{N}; m$ fest
			$$m+0\coloneqq m$$
			$$m + s(n) \coloneqq s(m + n)$$
			(rekursive (induktive) Definition für $m + n$)
			$$m \cdot 0\coloneqq 0$$
			$$m \cdot s(n) \coloneqq m + s(m + n)$$
	\end{description}

\subsection{Ganze Zahlen}
	\begin{description}
		\item[Motivation] $\mathbb{Z} \coloneqq \lbrace 0, 1, -1, 2, -2 \dots \rbrace$ (abzählbar)\\
			\begin{tabular}[t]{ll}
				$x+1=0$ & ist nicht lösbar in $\mathbb{N}$ \\
				$x+a=0$ & man nimmt zu jeder Zahl $a \in \mathbb{N}$ eine Gegenzahl $-a$
			\end{tabular} \\
			Lösung für $x+a=0$ (Ausnahme: $a=0$, denn $-0=0$)
		\item[Operationen] $+; \ -; \ \cdot$
		\item[spezielle Elemente] $0,\ 1$
		\item[lineare Ordnung] $<;\ \leq;\ >;\ \geq$
		\item[Gesetze] $(\forall a \in \mathbb{Z})$ gilt: \\
			\begin{tabular}{l|c|c}
				& Addition & Multiplikation \\ \hline
				& $a+0 = a$ & $a \cdot 1 = a$ \\ \hline
				Kommutativ & $a+b = b+a$ & $a \cdot b = b \cdot a $ \\ \hline
				Assoziativ & $(a+b)+c = a+(b+c)$ & $(a \cdot b) \cdot c = a \cdot (b \cdot c)$ \\ \hline
				& $a+(-a) = 0$
			\end{tabular} \\
			Ring-Identitäten: $a \cdot (b + c) = a \cdot b + a \cdot c$
		\item[Betrag] $|a| = \left\lbrace \begin{array}{rc} a & a \geq 0 \\ -a & a < 0\end{array} \right.$
		\item[Division] Es sein $a;m \in \mathbb{Z}|m \geq 1$ dann gibt es $q \in \mathbb{Z}$ mit $a=q \cdot m + r$ und $0 \leq r < m$. $q;r$ sind eindeutig bestimmt
	\end{description}

\subsection{Primzahlen}
	\begin{description}
		\item[Teiler] $a;b \in \mathbb{Z}$\\
			$a$ ist ein Teiler von $b$, geschrieben $a \mid c$, falls $(\exists c \in \mathbb{Z})\quad a \cdot c = b$\\
			Jede ganze Zahl $b$ ist teilbar durch: 1, -1, b , -b. Diese heißen die trivialen Teiler von $b$. Eigenschaften: \\
			$a \mid 0;\, a \mid 0$ \\
			$a \mid b \wedge b \mid c \Rightarrow a \mid b$ \\
			$a \mid b \Rightarrow a \mid (-b), (-a) \mid b, (-a) \mid (-b)$ \\
			$a;b \geq 1 \wedge a \mid b \Rightarrow a \leq b$
		\item[Primzahl] Eigenschaften:
			\begin{itemize}
				\item Eine ganze Zahl $p \in \mathbb{Z}$ heißt Primzahl, wenn $p \geq 2$ und $p$ nur triviale Teiler hat.
				\item Jede ganze Zahl $b \geq 2$ hat mindesten einen Primitiver.
				\item Es gibt unendlich viele Primzahlen. Beweis durch Widerspruch
					$$|\mathbb{P}| \in \mathbb{N}$$
					$n$ sei die Anzahl aller Primzahl, und alle Primzahlen seien in der Menge $\mathbb{P} = \lbrace p_1;p_2;p_3;\dots ; p_n \rbrace$. Man bilde $b = \prod\limits_{p \in \mathbb{ P}} + 1$. Dann ist $b \geq 2$ und laut Hilfssatz hat $b$ einen Primteiler, dieser sei $q$. Damit hat man eine Primzahl $q \not \in \mathbb{P}$ gefunden. Daraus folgt, dass die Konstruktion $\mathbb{P} = \lbrace p_1; \dots; p_n \rbrace | n \in \mathbb{N}$ nicht alle Primzahlen enthalten kann.
				\item Der kleinste Teiler einer Zahl $b \in \mathbb{N}|b \geq 2$ ist eine Primzahl.
			\end{itemize}
		\item[Fundamentalsatz der Arithmetik] Jede Zahl $b \geq 2$ lässt sich als Produktion von Primzahlen darstellen (Primfaktorisierung). Vorkommende Primzahlen und ihre Anzahl sind bis auf Reihenfolge eindeutig bestimmt.
	\end{description}
\subsection{Teilbarkeit}
$a \in \mathbb{Z},\ a \geq 2,\ a=(z_{n-1}z_{n-2} \dots z_1z_0)$ \\
\begin{tabular}{rcl}
	2 & $\Leftrightarrow$ & $z_0$ gerade \\
	3 & $\Leftrightarrow$ & Quersumme durch 3 teilbar \\
	4 & $\Leftrightarrow$ & $(z_1z_0)_{10}$ durch 4 teilbar \\
	5 & $\Leftrightarrow$ & $z_0 \in \lbrace 0; 1 \rbrace$ \\
	6 & $\Leftrightarrow$ & durch 2 und 3 teilbar \\
	7 & $\Leftrightarrow$ & ... \\
	8 & $\Leftrightarrow$ & $(z_2z_1z_0)_{(10)}$ durch 8 teilbar \\
	9 & $\Leftrightarrow$ & quersumme durch 9 teilbar \\
	10 & $\Leftrightarrow$ & durch 2 und 5 teilbar bzw. $z_0=0$
\end{tabular}

\subsection{Additionssysteme}
	''Strichliste (mit Abkürzungen)'' \\
	Z.B.: $5 = ||||| = \cancel{||||}$ oder römische Ziffern: \\
	\begin{tabular}[t]{|c|c|c|c|c|c|c|c|} \hline
		Großbuchstaben & \rom{1} & \rom{5} & \rom{10} & \rom{50} & \rom{100} & \rom{500} & \rom{1000} \\ \hline
		Wert & 1 & 5 & 10 & 50 & 100 & 500 & 1000 \\ \hline
	\end{tabular}
	
\subsection{Positionssysteme}
	\begin{itemize}
		\item Basis $B$, $B \in \mathbb{N}$, $B>=2$
		\item Ziffern für $0$ bis $B-1$. Jede Ziffer ein Zeichen.
		\item Zahl $= \dots z_2B^2+z_1B^1+z_0B^0+z_{-1}B^{-1} \dots$
	\end{itemize}
	
	\subsubsection{Umrechnung}
		\begin{description}
			\item[Polynom] $(z_{n-1}B^{n-1}z_{n-2}B^{n-2} \dots z_1B^{1}z_0B^{0})_{(B)}$
			\item[zu kleinere Basis] Fortgesetzte ganzzahlige Division mit Rest
				$217_{(10)}$ zur Basis 3 \\
				\begin{tabular}{r c}
					217 & 1 \\
					72 & 0 \\
					24 & 0 \\
					8 & 2 \\
					2 & 2 \\
					0 & 0
				\end{tabular} $217_{(10)} = 22001_{(3)}$
			\item[zu größerer Basis] mit Horner-Schema zum Dezimalsystem:
			
				\begin{tabular}{|c||c|c|c|c|c|} \hline
					Ziffern & 2 & 2 & 0 & 0 & 1 \\ \hline \hline
					$B=3$ & 0 & 6 & 24 & 72 & 216 \\ \hline
					& 2 & 8 & 24 & 72 & 217 \\ \hline
				\end{tabular} Addition $\downarrow$ dann Multiplikation $\nearrow$ mit $B$
				
				Wenn die Zielbasis eine Potenz der Ursprungsbasis ist, können $\log_{B_U}(B_Z)$ Stellen direkt zusammengefasst werden:
				$$(1000\ 0111\ 0001\ 1111)_{(2)}=(?)_{(16)}$$
				Hier können jeweils $\log_2(16)=4$ Stellen zusammengefasst werden:
				
				\begin{tabular}[t]{|l||c|c|c|c|} \hline
					$B = 2$ & 1000 & 0111 & 0001 & 1111 \\ \hline
					$B = 10$ & 8 & 7 & 1 & 15 \\ \hline
					$B = 16$ & 8 & 7 & 1 & F \\ \hline
				\end{tabular}		
		\end{description}
\end{document}