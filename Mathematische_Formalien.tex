\documentclass[12pt,a4paper]{article}
\usepackage[left=1.8cm, right=1.8cm, top = 2cm, bottom=2cm]{geometry}
\usepackage[german]{babel}
\usepackage{mathtools}
\usepackage{amsmath}
\usepackage{amsfonts}
\usepackage{amssymb}
\usepackage{wasysym} %Primille Symbol%
\usepackage{cancel} %durchgestrichen%
\usepackage{tabularx}
\usepackage{bookmark}
\usepackage{adjustbox}
\usepackage{pgf}
\usepackage{tikz}
\usepackage{pgfplots}
\pgfplotsset{compat = newest}
\usetikzlibrary{shapes}
\usepackage[]{hyperref}
\DeclareMathOperator\ggt{ggT}
\DeclareMathOperator\kgv{kgV}
\DeclareMathOperator\supri{sup}
\DeclareMathOperator\infim{inf}
\hypersetup{
    pdftitle = {HPI Bachelor Mathematik Vorkurs 2022},
    pdfauthor = {Yoan Tchorenev, Julian Hackenberg},
    bookmarksnumbered = true,
    bookmarksopen = true,
    bookmarksopenlevel = 3,
    colorlinks=false
}
\title{%
    HPI Bachelor Mathematik Vorkurs 2022 \\
    \large Wintersemester 2022/23 \\
}
\author{Yoan Tchorenev, Julian Hackenberg}
\newcommand{\rom}[1]{\uppercase\expandafter{\romannumeral #1 \relax}}

\setlength{\parindent}{0pt}
\setlength{\parskip}{10pt}
\begin{document}
    \begin{titlepage}
        \maketitle
        \tableofcontents
    \end{titlepage}


    \section{Logik}\label{sec:logik}
    \subsection{Begriffe}
\begin{description}
    \item[Aussage:] Eine Aussage ist eine Formel oder ein sprachliches Gebilde dem genau ein Wahrheitswert zugeordnet werden kann.

    \item[Warheitswerte] Genau der Eine oder der Andere \\
    \begin{tabular}{c|c}
        \textbf{F}alsch & \textbf{W}ahr \\
        0               & 1             \\
        $\bot$          & $\top$        \\
        \textbf{L}ow    & \textbf{H}igh
    \end{tabular}
    \item[Aussagevariable] A,B,C etc.\ stehen für eine Aussage
    \item[Junktoren] (Verknüpfer)
    \begin{description}
        \item[Negation] $\neg A$ \, ''nicht'', ''NOT'', auch: $A$, $\bar{A}$, $A'$ \\
        \begin{tabularx}{\linewidth}{c|X}
            \begin{tabular}[t]{c|c}
                A & $\neg A$ \\ \hline
                0 & 1        \\
                1 & 0
            \end{tabular} &
            Mathematisch: $\neg A = (A + 1) \bmod 2$ \\ \hline
        \end{tabularx}
        \item[Konjunktion] $A\wedge B$ \, ''A und B'', ''AND'', auch $A\cdot B$, AB \\
        \begin{tabularx}{\linewidth}{c|X}
            \begin{tabular}[t]{c|c||c}
                A & B & $A \wedge B$ \\ \hline\hline
                0 & 0 & 0            \\ \hline
                0 & 1 & 0            \\ \hline
                1 & 0 & 0            \\ \hline
                1 & 1 & 1
            \end{tabular} &
            \begin{tabular}[t]{ll}
                Mathematisch:               & $A \wedge B = A \cdot B$                             \\
                Kommutativ:                 & $A \wedge B \equiv B \wedge A$                       \\
                Assoziativ:                 & $A \wedge (B \wedge C) \equiv (A \wedge B) \wedge C$ \\
                Idempotent:                 & $A \wedge A \equiv A$                                \\
                $A \wedge \bot \equiv \bot$ & $A \wedge \top \equiv A$
            \end{tabular} \\ \hline
        \end{tabularx}
        \item[Disjunktion] $A\vee B$ \, ''A oder B'' (inklusiv), ''OR'' \\
        \begin{tabularx}{\linewidth}{c|X}
            \begin{tabular}[t]{c|c||c}
                A & B & $A \vee B$ \\ \hline\hline
                0 & 0 & 0          \\ \hline
                0 & 1 & 1          \\ \hline
                1 & 0 & 1          \\ \hline
                1 & 1 & 1
            \end{tabular} &
            \begin{tabular}[t]{ll}
                Mathematisch:           & $A \vee B = \min(A+B;1)$                     \\
                Kommutativ:             & $A \vee B \equiv B \vee A$                   \\
                Assoziativ:             & $A \vee (B \vee C) \equiv (A \vee B) \vee C$ \\
                Idempotent:             & $A \vee A \equiv A$                          \\
                $A \vee \bot \equiv A $ & $A \vee \top \equiv \top$
            \end{tabular} \\ \hline
        \end{tabularx}
        \item[Kontravalenz] $A\dot{\vee}B$ \, ''entweder A, oder B'' (exclusiv), ''XOR'', auch: $A\oplus B$ \\
        \begin{tabularx}{\linewidth}{c|X}
            \begin{tabular}[t]{c|c||c}
                A & B & $A \dot{\vee} B$ \\ \hline\hline
                0 & 0 & 0                \\ \hline
                0 & 1 & 1                \\ \hline
                1 & 0 & 1                \\ \hline
                1 & 1 & 0
            \end{tabular} &
            \begin{tabular}[t]{ll}
                Mathematisch:                & $A \dot{\vee} B = (A + B)\bmod 2$                                    \\
                Kommutativ:                  & $A \dot{\vee} B \equiv B \dot{\vee} A$                               \\
                Assoziativ:                  & $A \dot{\vee} (B \dot{\vee} C) \equiv (A \dot{\vee} B) \dot{\vee} C$ \\
                $\neg$ Idempotent:           & $A \dot{\vee} A \equiv \bot$                                         \\
                $A \dot{\vee} \bot \equiv A$ & $A \dot{\vee} \top \equiv \neg A$
            \end{tabular} \\ \hline
        \end{tabularx}
        \item[Konditional] $A\Rightarrow B$ ''wenn A dann B'' auch ''Subjunktion'', ''Implikation'', ''IMPLY'' \\
        \begin{tabularx}{\linewidth}{c|c|X}
            \begin{tabular}[t]{c|c||c}
                A & B & $A \Rightarrow B$ \\ \hline\hline
                0 & 0 & 1                 \\ \hline
                0 & 1 & 1                 \\ \hline
                1 & 0 & 0                 \\ \hline
                1 & 1 & 1
            \end{tabular} &
            \begin{tabular}[t]{c|c}
                A             & B          \\ \hline
                Prämisse      & Konklusion \\
                Voraussetzung & Konsequenz \\
                hinreichende  & notwendige
            \end{tabular}      &
            $A \Rightarrow B \equiv \neg A \vee B$ \newline
            Mathematisch: $A \Rightarrow B = \newline \min((A + 1) \bmod 2 + B; 1)$ \\ \hline
        \end{tabularx} \\
        \begin{tabular}[t]{rl}
            Eigenschaften    & $A \Rightarrow \bot \equiv \neg A; \quad A \Rightarrow \top \equiv \top; \quad \bot \Rightarrow A \equiv \top; \quad \top \Rightarrow A \equiv A$ \\
            Kontraposition   & $A \Rightarrow B \equiv \neg B \Rightarrow \neg A$                                                                                                \\
            Abtrennungsregel & $(A \wedge (A \Rightarrow B)) \Rightarrow B$                                                                                                      \\
            Kettenschluss    & $((A \Rightarrow B) \wedge (B \Rightarrow C)) \Rightarrow (A \Rightarrow C)$
        \end{tabular}

        \item[Bikonditional] $A \Leftrightarrow B$ ''A genau dann, wenn B'', ''XNOR'', auch ''Aquivalenz'' $\equiv$ \\
        \begin{tabularx}{\linewidth}{c|X}
            \begin{tabular}[t]{c | c || c}
                A & B & $A \Leftrightarrow B$ \\ \hline\hline
                0 & 0 & 1                     \\ \hline
                0 & 1 & 0                     \\ \hline
                1 & 0 & 0                     \\ \hline
                1 & 1 & 1
            \end{tabular} &
            \begin{tabular}[t]{ll}
                Mathematisch:                          & $A \Leftrightarrow B = (A + B + 1)\bmod 2$                                               \\
                Kommutativ:                            & $A \Leftrightarrow B \equiv B \Leftrightarrow A$                                         \\
                Assoziativ:                            & $A \Leftrightarrow (B \Leftrightarrow C) \equiv (A \Leftrightarrow B) \Leftrightarrow C$ \\
                $\neg$ Idempotent:                     & $A \Leftrightarrow A \equiv \top$                                                        \\
                $A \Leftrightarrow \bot \equiv \neg A$ & $A \Leftrightarrow \top \equiv A$
            \end{tabular} \\ \hline
        \end{tabularx}
    \end{description}
\end{description}


\subsection{Terme}
\begin{description}
    \item[Tautologie] Ein Term W heißt Tautologie, wenn er nur den Wahrheitswert 1 hat.
    \item[Äquivalenz] Zwei aussagenlogische Terme W und V heißen logisch äquivalent \[W \equiv V\] wenn sie gleichen Wahrheitswert haben.
    Zwei Terme W und V sind genau dann logisch äquivalent, wenn der Term $W\Leftrightarrow V$ Tautologie ist.
    \item[Klammern] Regeln:
    \begin{itemize}
        \item Außenklammern können weggelassen werden
        \item Die stärke der Zeichen ist konventionell: $\neg > \wedge > \vee$.
        D.h.:
        \[\neg A \vee B \wedge C \equiv (\neg A) \vee (B \wedge C)\]
        \item $\wedge$ und $\vee$ sind distributiv zueinander:
        \begin{gather*}
            A \wedge (A \vee C) \equiv (A \wedge B) \vee (A \wedge C)\\
            A \vee (A \wedge C) \equiv (A \vee B) \wedge (A \vee C)\\
        \end{gather*}
        \item $\wedge$ ist distributiv über $\dot{\vee}$:
        \[A \wedge (B \dot{\vee} C) \equiv (A \wedge B) \dot{\vee} (A \wedge C)\]
    \end{itemize}
    \item[De-Morganische Gesetze] \begin{gather*}
                                      \overline{A \wedge B} \equiv \overline{A} \vee \overline{B}\\
                                      \overline{A \vee B} \equiv \overline{A} \wedge \overline{B}\\
    \end{gather*}
\end{description}


\subsection{Beweise}
\begin{description}
    \item[Aussageform] Haben die Form einer Aussage, enthalte aber Variablen.
    $$3 + x = 5;\ A(x);\ B(x;y)$$
    \begin{itemize}
        \item werden zu Aussagen, wenn die Variablen belegt werden. Für die Variablen ist ein eingrenzender Grundbereich vorzugeben. Z.B.: $x \in \mathbb{N}$
        \item Wie Aussagen kann man Aussageformen miteinander Verknüpfen (mit Junktoren) und man erhält neue Aussageformen
    \end{itemize}
    \item[Quantoren] Außer der Belegung der Variablen mit Werten gibt es noch andere Möglichkeiten aus einer Aussageform eine Aussage zu machen. Ein Grundbereich $M$ muss vorgegeben sein.

    ''Für alle x aus $M$ gilt A(x)'' \newline
    Für alle $x \in \mathbb{N}$ gilt $3 + x = 5$ (falsche Aussage) kurz mit Allquantor $\forall$ :
    $$(\forall x \in \mathbb{N})\ 3 + x = 5$$

    ''Es existiert ein x aus $M$ mit A(x)'' \newline
    Es existiert (mindestens) ein $x \in \mathbb{N}$ mit $3 + x = 5$ (wahre Aussage) kurz mit Existenzquantor $\exists$ :
    $$(\exists x \in M)\ 3 + x = 5$$

    ''Es existiert höchsten ein x aus $M$ mit A(x)''
    $$(\forall x)(\forall y)\ (A(x) \wedge A(y) \Rightarrow x = y)$$

    ''Es existiert genau ein x aus $M$ mit A(x)''
    $$(\exists ! x) A(x) \equiv ((\exists x) A(x)) \wedge ((\forall x)(\forall y)\ (A(x) \wedge A(y) \Rightarrow x = y))$$
\end{description}




    \section{Mengenlehre}\label{sec:mengenlehre}
    \subsection{Begriffe}
Georg Cantor (1845-1918)
\begin{description}
    \item[Cantors naive Mengendefinition] Unter einer Menge verstehen wir eine Zusammenfassung von wohldefinierten Objekten $m$ unserer Anschauung oder unseres Denkens welche die Elemente von $M$ genannt werden, zu einem einheitlichen Ganzen.
    \item[Schreibweise]\
    \begin{itemize}
        \item $m \in M$ ($m$ ist Element von $M$)
        \item $m \not\in M$ ($m$ ist nicht Element von $M$, $\neg\ m \in M$)
    \end{itemize}
    \item[Mengendarstellung] verschiedene Möglichkeiten:
    \begin{itemize}
        \item allgemein mitels Eigenschaft $E(m)$ (Aussageform) $A=\lbrace m|E(m) \rbrace$ bzw.
        $$A = \lbrace m \in M | E(m) \rbrace = \lbrace m | m \in M \wedge E(m) \rbrace$$
        \item explizit für Menge mit wenigen endlich vielen Elementen:
        $$A=\lbrace a, b, c\rbrace$$
    \end{itemize}
    \item[Problem] Man darf nicht alle möglichen Zusammenfassungen bilden. Z.B.: die Menge aller Mengen die sich nicht selbst enthalten:
    $$R=\lbrace M | M \not \in M \rbrace$$
    $$R \in R \Leftrightarrow R \not \in R \equiv \bot$$
    \item[Lösung] Axiomatischer Aufbau der Mengenlehre
    \begin{description}
        \item[Extensionalitätsaxiom] Zwei Mengen A und B sind genau dann gleich, wenn sie die selben Elemente haben:
        $$A = B \Leftrightarrow (\forall x)(x \in A \Leftrightarrow x \in B)$$
        \item[Leere Menge] $\emptyset = \lbrace x | x \not = x\rbrace = \lbrace\rbrace$
        \item[Einermenge] $A=\lbrace a \rbrace$, $A = \lbrace x | x = a \rbrace$, $A \not = a$
        \item[Zweiermenge] $A=\lbrace a; b \rbrace$, $A = \lbrace x|(x=a \vee x=b) \wedge a \not = b \rbrace$
        \item[andere Mengen] \
        \begin{itemize}
            \item $\mathbb{N} = \lbrace 0;1;2;3;\dots \rbrace$ natürliche Zahlen
            \item $\mathbb{Z} = \lbrace \dots; (-1);0;1;\dots \rbrace$ ganze Zahlen
            \item $\mathbb{Q}$ rationale Zahlen
            \item $\mathbb{R}$ reelle Zahlen
            \item $\mathbb{C}$ komplexe Zahlen
        \end{itemize}
    \end{description}
    \item[Betrag] Anzahl der Elemente in der Menge (bei endlichen Mengen)
    \item[Teilmenge] $A \subseteq B \Leftrightarrow (\forall x)(x \in A \Rightarrow x \in B)$
    $$A \subseteq B \wedge B \subseteq C \Rightarrow A \subseteq C$$
    $$A \subseteq B \wedge B \subseteq A \Rightarrow A = B$$
    \item[Echte Teilmenge] $A \subset B$ oder $A \subsetneqq B \Leftrightarrow (\forall x)(x \in A \Rightarrow x \in B) \wedge A \not = B $
    \item[disjunkt] Die Mengen $A$ und $B$ heißen disjunkt (elementfremd) wenn: $A \cap B = \emptyset$
    \item[Kardinalität] Mächtigkeit
    \begin{description}
        \item[gleichmächtig] Zwei Mengen $A;B$ heißen gleich mächtig, wenn es eine bijektive Funktion $f : A \longrightarrow B$ gibt.
        $$A \sim B \Leftrightarrow (\exists f : A \longrightarrow B)$$
        $$A \sim B \wedge B \sim C \Rightarrow A \sim C$$
        \item[endlich] Menge $A$ heißt endlich, wenn $|A| \in \mathbb{N}$
        \item[abzählbar unendlich ] Eine Menge $A$ heißt abzählbar unendlich, wenn $$\mathbb{N} \sim A \wedge \exists f : \mathbb{N} \longrightarrow A\ \textrm{(bijektiv)}$$
        \item[nicht abzählbar unendlich] Meine Menge heißt nicht abzählbar unendlich, wenn sie weder endlich noch abzählbar unendlich ist.
        \item[Potenzmengen]$M \not \sim \mathcal{P}(M)$\\ Beweis:
        Angenommen es gäbe eine bijektive Funktion $f : A \longrightarrow \mathcal{P}(M)$ und
        $$A = \lbrace x \in M | x \not \in f(x) \rbrace \subset M$$
        Wir nehmen an dass $(\exists x \in M)\ f(x) = A$
        \begin{itemize}
            \item wenn $x \in f(x) $ dann $x \not \in A$ wegen $x \not \in f(x)$. Widerspruch da: $x \not \in A = x \not \in f(x)$
            \item wenn $x \not \in f(x)$ dann $x \in A$ wegen $x \in M$. Widerspruch da: $x \not \in A = x \not \in f(x)$
        \end{itemize}
    \end{description}
\end{description}
\subsection{Operationen auf Mengen}
\begin{description}
    \item[Vereinigung] $A \cup B = \lbrace x | x \in A \vee x \in B \rbrace$ \\
    \begin{tabular}{l|l|l}
        \adjustbox{valign = t}{
            \begin{tikzpicture}[thick, set/.style = {circle, minimum size = 2cm, fill=red}]
                \node [set, label={90:$A$}] (A) at (-0.5,0) {};
                \node [set, label={90:$B$}] (B) at (0.5,0) {};
                \draw (-0.5,0) circle(1);
                \draw (0.5,0) circle(1);
            \end{tikzpicture}
        }                                                               &
        \adjustbox{valign = t}{
            \begin{tikzpicture}[thick, set/.style = {circle, minimum size = 2cm, draw = black, fill=red}]
                \node [set, label={90:$A$}] (A) at (-1.1,0) {};
                \node [set, label={90:$B$}] (B) at (1.1,0) {};
            \end{tikzpicture}} &
        $\begin{array}{r c l}
             |A \cup B| & = & |A| + |B \setminus A| \\
             & = & |B| + |A \setminus B|
        \end{array}$
    \end{tabular}
    \item[Durchschnitt] $A \cap B := \lbrace x | x \in A \wedge x \in B \rbrace$ \\
    \begin{tabular}{l|l|l}
        \adjustbox{valign = t}{
            \begin{tikzpicture}[thick, set/.style = {circle, minimum size = 2cm, draw = black}]
                \begin{scope}
                    \clip (-0.5,0) circle(1);
                    \fill[red] (0.5, 0) circle (1);
                \end{scope}
                \node [set, label={90:$A$}] (A) at (-0.5,0) {};
                \node [set, label={90:$B$}] (B) at (0.5,0) {};
            \end{tikzpicture}
        } &
        \adjustbox{valign = t}{
            \begin{tikzpicture}[thick, set/.style = {circle, minimum size = 2cm, draw = black}]
                \node [set, label={90:$A$}] (A) at (-1.1,0) {};
                \node [set, label={90:$B$}] (B) at (1.1,0) {};
            \end{tikzpicture}
        } &
        $\begin{array}{r c l}
             |A \cap B| & = & |A| - |A \setminus B| \\
             & = & |B| - |B \setminus A|
        \end{array}$
    \end{tabular}
    \item[Mengendifferenz] $A \setminus B = \lbrace x | x \in A \wedge x \not \in B \rbrace$ \\
    \begin{tabular}{l|l|l}
        \adjustbox{valign = t}{
            \begin{tikzpicture}[thick, set/.style = {circle, minimum size = 2cm, draw = black}]
                \begin{scope} [even odd rule]
                    \clip (-0.5,0) circle(1) (0.5,0) circle(1);1
                    \fill [red] (-0.5,0) circle (1);
                \end{scope}
                \node [set, label={90:$A$}] (A) at (-0.5,0) {};
                \node [set, label={90:$B$}] (B) at (0.5,0) {};
            \end{tikzpicture}
        } &
        \adjustbox{valign = t}{
            \begin{tikzpicture}[baseline=(current bounding box.north), thick, set/.style = {circle, minimum size = 2cm, draw = black}]
            \node [set, fill = red, label={90:$A$}] (A) at (-1.1,0) {};
            \node [set, label={90:$B$}] (B) at (1.1,0) {};
            \end{tikzpicture}
        } &
    \end{tabular}
    \item[symmetrische Differenz] $A \Delta B = (A \setminus B) \cup (B \setminus A)$ \\
    \begin{tabular}{l|l|l}
        \adjustbox{valign = t}{
            \begin{tikzpicture}[thick, set/.style = {circle, minimum size = 2cm, draw = black}]
                \fill [even odd rule, red] (-0.5,0) circle (1) (0.5,0) circle (1);
                \node [set, label={90:$A$}] (A) at (-0.5,0) {};
                \node [set, label={90:$B$}] (B) at (0.5,0) {};
            \end{tikzpicture}
        } &
        \adjustbox{valign = t}{
            \begin{tikzpicture}[baseline=(current bounding box.north), thick, set/.style = {circle, minimum size = 2cm, draw = black, fill = red}]
            \node [set, label={90:$A$}] (A) at (-1.1,0) {};
            \node [set, label={90:$B$}] (B) at (1.1,0) {};
            \end{tikzpicture}
        } &
        $|A \Delta B| = |A \setminus B| + |B \setminus A|$
    \end{tabular}
    \item[Potenzmengen] $\mathcal{P}(A) \coloneqq \lbrace B|B \subseteq A \rbrace;\ |\mathcal{P}(A)| = 2^{|A|}$
    \item[ungeordnets Paar] $\lbrace a,b \rbrace = \lbrace c,d \rbrace \Rightarrow (a=c \wedge b=d) \vee (a=d \wedge b=c)$
    \item[geordnetes Paar] $\lbrace a,b \rbrace = \lbrace c,d \rbrace \Rightarrow a=c \wedge b=d$ (Das geht!)
    \item[Mengenprodukt] $A \times B = \lbrace (a,b)|a \in A \wedge b \in B \rbrace$ (nicht Kommutativ, (strenggenommen) nicht assoziativ)
    \begin{align*}
    (A \times B) \times C & \not = A \times (B \times C) \\
    ((a,b),c)             & \not = (a,(b,c))
    \end{align*}
    Gegeben sein
    $$A = \lbrace 1, 2 \rbrace$$
    $$B = \lbrace a, b, c \rbrace$$
    dann ist:
    $$A \times B = \lbrace (1,a),(2,a)(1,b)(2,b),(1,c)(2,c) \rbrace$$ \\
    $|A \times B| = |A| \cdot |B|$
\end{description}



    \section{Funktionen}\label{sec:funktionen}
    Funktionen sind im wesentlich Zuordnungen.

\subsection{Begriffe}\label{subsec:funktionen_begriffe}
\begin{description}
    \item[Definition] Zur Definition einer Funktion $f$ braucht man drei Dinge
    \begin{itemize}
        \item Menge $A$, der Definitionsbereich von $f$, $A = D_f$
        \item Menge $B$, der Wertevorrat von $f$, $B = W_f$
        \item Eine Zuordnung, die jedem $a \in A$ genau ein Element $b \in B$ zuordnet \\
        Schreibweise: $b = f(a)$ bzw. $a \longmapsto f(a)$ \\
        Mathematisch wird diese Zuordnung gegeben durch eine Menge von geordneten Paaren
        \[\textrm{Graph}(f) = \lbrace(a,f(a)) | a \in A \rbrace \subseteq A \times B\]
        mit den Eigenschaften:
        \begin{itemize}
            \item $(\forall a \in A)(\exists b \in B)\ (a;b) \in \textrm{Graph}(f)$ (Vollständigkeit)
            \item $(\forall a \in A)(\forall b_1,b_2 \in B)\ (a;b_1);(a;b_2) \in \textrm{Graph}(f)\Rightarrow b_1 = b_2$ (Eindeutigkeit)
        \end{itemize}
    \end{itemize}
    \item[Schreibweise]
    \begin{alignat*}{3}
        f :\  & A \longrightarrow &  & B \quad , \quad a &  & \longmapsto f(a) = \cdots \\
        & D_f               &  & W_v               &  & \textrm{Graph}
    \end{alignat*}
    \item[Bild] Die Menge aller Funktionswerte von $f$. $\lbrace f(a) | a \in A \rbrace = \lbrace b \in B  | (\exists a \in A) b = f(a) \rbrace \subseteq B$
    \item[surjektiv] $(\forall b \in B)(\exists a \in A) \ f(a) = b$ \\
    \begin{tabularx}{\linewidth}{l|X}
        \adjustbox{valign = t}{
            \begin{tikzpicture}[thick, set/.style = {ellipse, minimum width = 2cm, minimum height = 4cm, draw = black, align = center}, element/.style = {circle, draw = black, minimum size = 0.7, outer sep = 0.05cm}]
                \node [set, label={90:$A$}] (A) at (-1.5,0) {};
                \node [set, label={90:$B$}] (B) at (1.5,0) {};
                \node [element] (1) at (-1.5, 1.5) {1};
                \node [element] (2) at (-1.5, 0.5) {2};
                \node [element] (3) at (-1.5, -0.5) {3};
                \node [element] (4) at (-1.5, -1.5) {4};
                \node [element] (A) at (1.5, 1.5) {A};
                \node [element] (B) at (1.5, 0.5) {B};
                \node [element] (C) at (1.5, -0.5) {C};
                \draw [->] (1) to (A);
                \draw [->] (2) to (B);
                \draw [->] (3) to (C);
                \draw [->] (4) to (C);
            \end{tikzpicture}
        } &
        Für jedes Element in $B$ existiert (mindestens) ein Urbild in $A$.
        Für jede rein surjektive Abbildung gilt:
        \[|A|>|B|\] \\ \hline
    \end{tabularx}
    \item[injektiv] $(\forall a_1,a_2 \in A) (a_1 \not = a_2 \Rightarrow f(a_1) \not = f(a_2))$ \\
    \begin{tabularx}{\linewidth}{l|X}
        \adjustbox{valign = t}{
            \begin{tikzpicture}[thick, set/.style = {ellipse, minimum width = 2cm, minimum height = 4cm, draw = black, align = center}, element/.style = {circle, draw = black, minimum size = 0.7, outer sep = 0.05cm}]
                \node [set, label={90:$A$}] (A) at (-1.5,0) {};
                \node [set, label={90:$B$}] (B) at (1.5,0) {};
                \node [element] (1) at (-1.5, 1.5) {1};
                \node [element] (2) at (-1.5, 0.5) {2};
                \node [element] (3) at (-1.5, -0.5) {3};
                \node [element] (A) at (1.5, 1.5) {A};
                \node [element] (B) at (1.5, 0.5) {B};
                \node [element] (C) at (1.5, -0.5) {C};
                \node [element] (D) at (1.5, -1.5) {D};
                \draw [->] (1) to (A);
                \draw [->] (2) to (B);
                \draw [->] (3) to (D);
            \end{tikzpicture}
        } &
        Für jede zwei Elemente in $A$ gilt, dass wenn sie verschieden von einander sind, dann auch ihre Funktionswerte von $f$ verschieden sind.
        Also hat jedes Element in $B$ höchstens ein Urbild.
        Für jede rein injektive Abbildung gilt:
        \[|A|<|B|\] \\ \hline
    \end{tabularx}
    \item[bijektiv]  surjektiv $\wedge$ injektiv: $(\forall b \in B)(\exists ! a \in A)\ f(a) = b$ \\
    \begin{tabularx}{\linewidth}{l|X}
        \adjustbox{valign = t}{
            \begin{tikzpicture}[thick, set/.style = {ellipse, minimum width = 2cm, minimum height = 4cm, draw = black, align = center}, element/.style = {circle, draw = black, minimum size = 0.7, outer sep = 0.05cm}]
                \node [set, label={90:$A$}] (A) at (-1.5,0) {};
                \node [set, label={90:$B$}] (B) at (1.5,0) {};
                \node [element] (1) at (-1.5, 1.5) {1};
                \node [element] (2) at (-1.5, 0.5) {2};
                \node [element] (3) at (-1.5, -0.5) {3};
                \node [element] (4) at (-1.5, -1.5) {4};
                \node [element] (A) at (1.5, 1.5) {A};
                \node [element] (B) at (1.5, 0.5) {B};
                \node [element] (C) at (1.5, -0.5) {C};
                \node [element] (D) at (1.5, -1.5) {D};
                \draw [->] (1) to (A);
                \draw [->] (2) to (B);
                \draw [->] (3) to (C);
                \draw [->] (4) to (D);
            \end{tikzpicture}
        } &
        Für jedes Element in $B$ existiert genau ein Urbild in $A$.
        Für jede bijektive Abbildung gilt:
        \[|A|=|B|\] \\ \hline
    \end{tabularx}
    \item[Identitätsfunktion] $id_A : A \longrightarrow A , a \longmapsto a$ z.B. $f(x) = x$
    \item[Komposition] $f : A \longrightarrow B;\ g : B \longrightarrow C$
    \[(g \circ f) : A \longrightarrow C, a \longmapsto g(f(a))\]
    \begin{alignat*}{3}
        f : A \longrightarrow B \Rightarrow & f    &  & = f \circ id_A &  & = id_A \circ f \\
        & f(a) &  & = f(id_A(a))   &  & = id_A(f(a))
    \end{alignat*}
\end{description}


\subsection{Umkehrfunktion}\label{subsec:funktionen_umkehrfunktion}
\begin{description}
    \item[Umkehrbarkeit] (im engeren sinne) $f : A \longrightarrow B$
    $$\leftrightarrow (\exists g : B \longrightarrow A) g \circ f = id_A \wedge f \cdot g = id_B$$
    $$(\forall a \in A)\ g(f(a)) = a$$
    $$(\forall b \in B)\ f(g(b)) = b$$
    Die Funktion $g : B \longrightarrow A$ heißt dann Umkehrfunktion von f, geschrieben $g = f^{-1}$.
    $$f^{-1} \not = (f)^{-1}$$
    Satz: Eine Funktion $f : A \longrightarrow B$ ist genau dann umkehrbar (i.e.s), wenn sie bijektiv ist.
    \item[Umkehrbarkeit in der Analysis] Eine Funktion $f : A \longrightarrow B$ heißt Umkehrbar, wenn die zugehörige Funktion $f : A \longrightarrow \textrm{Bild}(f)$ umkehrbar ist.
    Satz: Eine Funktion $f : A \longrightarrow B$ ist genau dann umkehrbar (i.w.s), wenn sie injektiv ist.
\end{description}

\subsubsection{Potenzfunktion}
\[f : \mathbb{R} \longrightarrow \mathbb{R},\ x \longmapsto x^n\]
\begin{description}
    \item[quadratisch] \
    \begin{tabular}[t]{cc}
        $f : \mathbb{R} \longrightarrow \mathbb{R},\ x \longmapsto x^2$ & $f^{*-1} : \mathbb{R}_0^+ \longrightarrow \mathbb{R},\ x \longmapsto \sqrt{x}$ \\
        \begin{tikzpicture}
            \begin{axis}
                [
                x = 1cm, y = 1cm,
                xmin = -2, xmax = 2,
                ymin = -1, ymax = 4,
                axis lines = center,
                xtick={-1,0,...,1},
                ytick={0,1,...,3},
                xlabel={$x$},
                ylabel={$y$},
                xlabel style={below right},
                ylabel style={above left},
                grid=both]
                \addplot[
                    domain = -2:2,
                    samples = 200,
                    smooth,
                    thick,
                    blue,
                ] {x^2};
            \end{axis}
        \end{tikzpicture}                                               &
        \begin{tikzpicture}
            \begin{axis}
                [
                x = 1cm, y = 1cm,
                xmin = -2, xmax = 4,
                ymin = -1, ymax = 4,
                axis lines = center,
                xtick={-1,0,...,3},
                ytick={0,1,...,3},
                xlabel={$x$},
                ylabel={$y$},
                xlabel style={below right},
                ylabel style={above left},
                grid=both]
                \addplot[
                    domain = 0:4,
                    samples = 200,
                    smooth,
                    thick,
                    blue,
                ] {sqrt(x)};
            \end{axis}
        \end{tikzpicture}
    \end{tabular}
    \item[kubisch] \
    \begin{tabular}[t]{cc}
        $f : \mathbb{R} \longrightarrow \mathbb{R},\ x \longmapsto x^3$ & $f^{-1} : \mathbb{R} \longrightarrow \mathbb{R},\ x \longmapsto \sqrt[3]{x}$ \\
        \begin{tikzpicture}
            \begin{axis}
                [
                x = 1cm, y = 1cm,
                xmin = -2, xmax = 2,
                ymin = -3, ymax = 3,
                axis lines = center,
                xtick={-1,0,...,1},
                ytick={-2,-1,...,2},
                xlabel={$x$},
                ylabel={$y$},
                xlabel style={below right},
                ylabel style={above left},
                grid=both]
                \addplot[
                    domain = -2:2,
                    samples = 200,
                    smooth,
                    thick,
                    blue,
                ] {x^3};
            \end{axis}
        \end{tikzpicture}                                               &
        \begin{tikzpicture}
            \begin{axis}
                [
                x = 1cm, y = 1cm,
                xmin = -4, xmax = 4,
                ymin = -3, ymax = 3,
                axis lines = center,
                xtick={-3,-2,...,3},
                ytick={-2,-1,...,2},
                xlabel={$x$},
                ylabel={$y$},
                xlabel style={below right},
                ylabel style={above left},
                grid=both]
                \addplot[
                    domain = -4:4,
                    samples = 200,
                    smooth,
                    thick,
                    blue,
                ] {sign(x) * abs(x)^(1/3)};
            \end{axis}
        \end{tikzpicture}
    \end{tabular}
\end{description}


\subsubsection{Exponentialfunktionen}
\[f : \mathbb{R} \longrightarrow \mathbb{R}^+,\ x \longmapsto b^x \ | b \in \mathbb{R}^+ \setminus \lbrace 1 \rbrace\]
\begin{tabular}[t]{cc}
    $f : \mathbb{R} \longrightarrow \mathbb{R}^+,\ x \longmapsto 2^x$ & $f*^{-1} : \mathbb{R}^+ \longrightarrow \mathbb{R},\ x \longmapsto \log_2(x)$ \\
    \begin{tikzpicture}
        %! suppress = Ellipsis
        \begin{axis}
            [
            x = 1cm, y = 1cm,
            xmin = -2, xmax = 2,
            ymin = -3, ymax = 3,
            axis lines = center,
            xtick={-1,0,...,1},
            ytick={-2,-1,...,2},
            xlabel={$x$},
            ylabel={$y$},
            xlabel style={below right},
            ylabel style={above left},
            grid=both]
            \addplot[
                domain = -2:2,
                samples = 200,
                smooth,
                thick,
                blue,
            ] {2^x};
        \end{axis}
    \end{tikzpicture}                                                 &
    \begin{tikzpicture}
        %! suppress = Ellipsis
        \begin{axis}
            [
            x = 1cm, y = 1cm,
            xmin = -2, xmax = 4,
            ymin = -3, ymax = 3,
            axis lines = center,
            xtick={-1,0,...,3},
            ytick={-2,-1,...,2},
            xlabel={$x$},
            ylabel={$y$},
            xlabel style={below right},
            ylabel style={above left},
            grid=both]
            \addplot[
                domain = -0.01:4,
                samples = 200,
                smooth,
                thick,
                blue,
            ] {ln(x)/ln(2)};
        \end{axis}
    \end{tikzpicture}
\end{tabular}





    \section{Zahlen}\label{sec:zahlen}
    \subsection{Sprachunterschiede}
\begin{tabular}{|l|l|l|} \hline
& deutsch   & US-Englisch \\ \hline \hline
$10^6$    & Million   & million     \\ \hline
$10^9$    & Milliarde & billion     \\ \hline
$10^{12}$ & Billion   & trillion    \\ \hline
$10^{15}$ & Billiarde & quadrillion \\ \hline
$10^{18}$ & Trillion  & quintillion \\ \hline
\end{tabular}

\subsection{natürliche Zahlen}
$\mathbb{N} = \lbrace 0, 1, 2, 3, \dots \rbrace$
\begin{description}
    \item[unendlichkeits Axiom] Es gibt unendliche Mengen
    \item[Peano-Axiome] 5 Stück:
    \begin{itemize}
        \item $0 \in \mathbb{N}$, null ist eine natürliche Zahl
        \item es gibt eine Nachfolgerfunktion $s : \mathbb{N} \longrightarrow \mathbb{N}$
        \item $s$ ist injektiv
        \item $0 \not \in \textrm{Bild}(s)$, Null ist nicht Nachfolger einer natürlichen Zahl
        \item Für jede Menge $M \subseteq \mathbb{N}$ gilt:
        \[(0 \in \mathbb{N} \wedge (\forall n \in \mathbb{N})(n \in M \Rightarrow s(n) \in M)) \Rightarrow M = \mathbb{N}\]
        Modifikation: steht $M \subseteq \mathbb{N}$ kann man das auch als Eigenschaft $E_M(n)$ ausdrücken.
        \[E_M(n) \Leftrightarrow n \in M\]
    \end{itemize}
    \item[Vollständige Induktion] am Beispiel für einen Beweis der Gaußschen Summenformel
    \begin{description}
        \item[Induktionsvoraussetzung] Die Annahme: $A(n) \Leftrightarrow 1 + 2 + \dots + n = \frac{n(n+1)}{2}$
        \item[Induktionsanfang] Der Beweis, dass der Anfang gültig ist: $A(1) = 1$
        \item[Induktionsbehauptung] Das Einsetzen von $(n + 1)$ für $n$:
        \[A(n + 1) \Leftrightarrow 1 + \dots + n + (n + 1)= \frac{(n + 1)((n + 1)+1)}{2}\]
        \item[Induktionsschritt] Zeigen, dass aus der Induktionsvoraussetzung
        \[A(n) \Leftrightarrow 1 + \dots + n = \frac{n(n+1)}{2}\]
        die Induktionsbehauptung
        \[A(n + 1) \Leftrightarrow 1 + \dots + n + (n + 1) = \frac{(n + 1)((n + 1)+1)}{2}\]
        folgt. In diesem speziellen Fall:
        \begin{align*}
            A(n + 1) \Leftrightarrow 1 + \dots + n + (n + 1) & = \frac{n(n + 1)}{2} + (n + 1)  \\
            & = \frac{n(n + 1) + 2(n + 1)}{2} \\
            & = \frac{(n + 1)(n + 2)}{2}      \\
            & = \frac{(n + 1)((n + 1)+1)}{2}
        \end{align*}
    \end{description}
    \item[Addition] $m \in \mathbb{N}; m$ fest
    \[m+0\coloneqq m\]
    \[m + s(n) \coloneqq s(m + n)\]
    (rekursive (induktive) Definition für $m + n$)
    \[m \cdot 0\coloneqq 0\]
    \[m \cdot s(n) \coloneqq m + s(m + n)\]
\end{description}


\subsection{Ganze Zahlen}
\begin{description}
    \item[Motivation] $\mathbb{Z} \coloneqq \lbrace 0, 1, -1, 2, -2 \dots \rbrace$ (abzählbar)\\
    \begin{tabular}[t]{ll}
        $x+1=0$ & ist nicht lösbar in $\mathbb{N}$                               \\
        $x+a=0$ & man nimmt zu jeder Zahl $a \in \mathbb{N}$ eine Gegenzahl $-a$
    \end{tabular} \\
    Lösung für $x+a=0$ (Ausnahme: $a=0$, denn $-0=0$)
    \item[Operationen] $+; \ -; \ \cdot$
    \item[spezielle Elemente] $0,\ 1$
    \item[lineare Ordnung] $<;\ \leq;\ >;\ \geq$
    \item[Gesetze] $(\forall a \in \mathbb{Z})$ gilt: \\
    \begin{tabular}{l|c|c}
        & Addition            & Multiplikation                              \\ \hline
        & $a+0 = a$           & $a \cdot 1 = a$                             \\ \hline
        Kommutativ & $a+b = b+a$         & $a \cdot b = b \cdot a $                    \\ \hline
        Assoziativ & $(a+b)+c = a+(b+c)$ & $(a \cdot b) \cdot c = a \cdot (b \cdot c)$ \\ \hline
        & $a+(-a) = 0$
    \end{tabular} \\
    Ring-Identitäten: $a \cdot (b + c) = a \cdot b + a \cdot c$
    \item[Betrag] $|a| = \left\lbrace \begin{array}{rc} a & a \geq 0 \\ -a & a < 0\end{array} \right.$
    \item[Division] Es sein $a;m \in \mathbb{Z}|m \geq 1$ dann gibt es $q \in \mathbb{Z}$ mit $a=q \cdot m + r$ und $0 \leq r < m$. $q;r$ sind eindeutig bestimmt
\end{description}

\subsection{Primzahlen}
\begin{description}
    \item[Teiler] $a;b \in \mathbb{Z}$\\
    $a$ ist ein Teiler von $b$, geschrieben $a \mid c$, falls $(\exists c \in \mathbb{Z})\quad a \cdot c = b$\\
    Jede ganze Zahl $b$ ist teilbar durch: 1, -1, b , -b.
    Diese heißen die trivialen Teiler von $b$.
    Eigenschaften: \\
    $a \mid 0;\, a \mid 0$ \\
    $a \mid b \wedge b \mid c \Rightarrow a \mid b$ \\
    $a \mid b \Rightarrow a \mid (-b), (-a) \mid b, (-a) \mid (-b)$ \\
    $a;b \geq 1 \wedge a \mid b \Rightarrow a \leq b$
    \item[Primzahl] Eigenschaften:
    \begin{itemize}
        \item Eine ganze Zahl $p \in \mathbb{Z}$ heißt Primzahl, wenn $p \geq 2$ und $p$ nur triviale Teiler hat.
        \item Jede ganze Zahl $b \geq 2$ hat mindesten einen Primitiver.
        \item Es gibt unendlich viele Primzahlen.
        Beweis durch Widerspruch
        \[|\mathbb{P}| \in \mathbb{N}\]
        $n$ sei die Anzahl aller Primzahl, und alle Primzahlen seien in der Menge $\mathbb{P}$.
        Man bilde $b = \prod\limits_{p \in \mathbb{ P}} + 1$.
        Dann ist $b \geq 2$ und laut Hilfssatz hat $b$ einen Primteiler, dieser sei $q$.
        Damit hat man eine Primzahl $q \not \in \mathbb{P}$ gefunden.
        Daraus folgt, dass die Konstruktion $\mathbb{P} = \lbrace p_1; \dots; p_n \rbrace | n \in \mathbb{N}$ nicht alle Primzahlen enthalten kann.
        \item Der kleinste Teiler einer Zahl $b \in \mathbb{N}|b \geq 2$ ist eine Primzahl.
        \item Der kleinste Primteiler $p$ einer Zahl $a \in \mathbb{Z};\ a \geq 2;\ a \not \in \mathbb{P}$ ist $p \leq \sqrt{a}$
    \end{itemize}
    \item[Fundamentalsatz der Arithmetik] Jede Zahl $b \geq 2$ lässt sich als Produktion von Primzahlen darstellen (Primfaktorisierung).
    Vorkommende Primzahlen und ihre Anzahl sind bis auf Reihenfolge eindeutig bestimmt.
\end{description}


\subsection{Teilbarkeit}
$a \in \mathbb{Z},\ a \geq 2,\ a=(z_{n-1}z_{n-2} \dots z_1 z_0)$ \\
\begin{tabular}{rcl}
    2  & $\Leftrightarrow$ & $z_0$ gerade                           \\
    3  & $\Leftrightarrow$ & Quersumme durch 3 teilbar              \\
    4  & $\Leftrightarrow$ & $(z_1 z_0)_{10}$ durch 4 teilbar       \\
    5  & $\Leftrightarrow$ & $z_0 \in \lbrace 0; 1 \rbrace$         \\
    6  & $\Leftrightarrow$ & durch 2 und 3 teilbar                  \\
    7  & $\Leftrightarrow$ & ...                                    \\
    8  & $\Leftrightarrow$ & $(z_2 z_1 z_0)_{(10)}$ durch 8 teilbar \\
    9  & $\Leftrightarrow$ & quersumme durch 9 teilbar              \\
    10 & $\Leftrightarrow$ & durch 2 und 5 teilbar bzw. $z_0=0$
\end{tabular}


\subsection{ggT und kgV}
Sein $a;b \in \mathbb{Z}$

Ein gemeinsamer Teiler von $a$ und $b$ ist eine Zahl $t \in \mathbb{N}$ mit $t \mid a$ und $t \mid b$.
Abkürzung: $\textrm{ggt}(a;b)$.

Ein gemeinsames vielfaches von $a$ und $b$ ist ein $s \in \mathbb{Z}$ mit $a \mid s$ und $b \mid s$.
Abkürzung: $\textrm{kgv}(a;b)$.

\subsubsection{mit Primfaktorisierung}
$a \dots p^m;\ a \dots p^n$

$\ggt(a;b)\quad p^{\min(m;n)};\ \kgv(a;b)\quad p^{\max(m;n)}$

$m + n = \min(m;n) + \max(m;n) \Rightarrow a \cdot b = \textrm{ggt}(a;b) \cdot \textrm{kgv}(a;b)$

$a = 5940 = 2^2 \cdot 3^3 \cdot 5 \cdot 11$

$b = 11760 = 2^4 \cdot 3 \cdot 5 \cdot 7$

\[\begin{array}{rcccccl}
          a = & 2^2 & \cdot 3^3 & \cdot 5^1 & \cdot 7^0 & \cdot 11^1 \\
          b = & 2^4 & \cdot 3^1 & \cdot 5^1 & \cdot 7^2 & \cdot 11^0 \\
          \ggt(a;b) = & 2^2 & \cdot 3^1 & \cdot 5^1 & \cdot 7^0 & \cdot 11^0 & = 60      \\
          \kgv(a;b) = & 2^4 & \cdot 3^3 & \cdot 5^1 & \cdot 7^2 & \cdot 11^1 & = 1164240
\end{array}\]


\subsubsection{Euklidischer Algorithmus}
\begin{itemize}
    \item Es sein $a_1;a_2 \in \mathbb{Z},\ a_1 > a_2 \geq 1$
    \item Division mit Rest: $a_1 = q_2 a_2 + a_3$ mit $0 \leq a_3 < a_2$
    \item Sei $g$ gem.\ Teiler von $a_1$ und $a_2$, $a_1 - q_2 \cdot a_2 = a_3 \Rightarrow g$ gem.\ Teiler von $a_2$ und $a_3$
    \item Sei $g$ gem.\ Teiler von $a_2$ und $a_3$, $a_1 = q_2 \cdot a_2 + a_3 \Rightarrow g$ gem.\ Teiler von $a_1$ und $a_2$
    \item $\Rightarrow \ggt(a_1;a_2) = \ggt(a_2;a_3)$
    \item $a_{n}=q_{n+1}a_{n+1} + 0 \Rightarrow \ggt(a_n;a_{n+1}) = \ggt(a_1,a_2) = a_{n+1}$
\end{itemize}
Beispiel: $\ggt(851,2183);\ a=2183;\ a_2 = 851$
\begin{alignat*}{3}
    2183 & = 2 &  & \cdot 851 &  & + 481 \\
    851  & = 1 &  & \cdot 481 &  & + 370 \\
    481  & = 1 &  & \cdot 370 &  & + 111 \\
    370  & = 3 &  & \cdot 111 &  & + 37  \\
    111  & = 3 &  & \cdot 37
\end{alignat*}
$\ggt(851;2183) = 37$
\begin{itemize}
    \item Es gibt die darstellung $\ggt(a_1;a_2) = s\cdot a_1 + t \cdot a_2$ mit $s;t \in \mathbb{Z}$
    \item $a;c \in \mathbb{Z}$ heißen Teilerfremd wenn $\ggt(a;b) = 1$
    \item Sei $t \mid a \cdot b$ und $a;t$ teilerfremd $\Rightarrow t \mid b$
    \item Sei $p \in \mathbb{P}$ und $p \mid a \cdot b \Rightarrow p \mid a \vee p \mid b$ denn:
    \begin{itemize}
        \item[Fall 1] $p \mid a$ Ausdruck wahr
        \item[Fall 2] $p \nmid a \Rightarrow \ggt(p;a) = 1$
    \end{itemize}
\end{itemize}



\subsection{Rationale Zahlen}
\begin{description}
    \item[Problem] $a \nmid a,\ a \cdot x = a$ nicht lösbar in $\mathbb{Z}$
    \item[Lösung] nehmen $\frac{a}{b}$ hinzu. $\mathbb{Q} \coloneqq \lbrace \frac{a}{b} | a;b \in \mathbb{Z} \wedge b \not = 0 \rbrace$ außerdem: $\frac{a}{b} = \frac{c}{d} \Leftrightarrow ad = bc$.
    Eine rationale Zahl entspricht also einer Menge von Brüchen, die als selbe Zahl, betrachtet werden.

    Also $\frac{a}{b} = \frac{a \cdot t}{b \cdot t}$ denn $a \cdot b \cdot t = a \cdot t \cdot b$.
    Jede rationale Zahl entspricht genau einem unkürzbaren Bruch $\frac{a}{b}$ mit $\ggt(a;b) = 1$ und $b \geq 1$.

    Einbettung: $\mathbb{Z}\ni z \longmapsto \frac{z}{1} \in \mathbb{Q}$, dann gilt $\mathbb{Z} \subset \mathbb{Q}$

    $\mathbb{Q}$ unendlich, $\mathbb{N} \sim \mathbb{Q}$, $\mathbb{Q}$ abzählbar: wir sortieren $a + b$ nach $\frac{a}{b}$
    \[\begin{array}{lccc}
          a + b = 1 &             &             & \frac{0}{1} \\
          a + b = 2 &             & \frac{0}{2} & \frac{1}{1} \\
          a + b = 3 & \frac{0}{3} & \frac{1}{2} & \frac{2}{1}
    \end{array}\]
    \item[Operationen]
    \begin{gather*}
        \frac{a}{b} + \frac{c}{d} \coloneqq \frac{ad+bc}{bd}\\
        \frac{a}{b} - \frac{c}{d} \coloneqq \frac{ad-bc}{bd}\\
        q = \frac{a}{b};\ a \not =;\ b \not 0\\
        q^{-1} = \frac{b}{a}\\
        q \cdot q^{-1} = \frac{ab}{ab} = 1\\
        \frac{c}{d}:\frac{a}{b}\coloneqq\frac{c}{d}  \left(\frac{a}{b}\right)^{-1}\\
    \end{gather*}
    Bruchstrich entspricht Division \\
    Division ist nicht assoziativ
    \[q:v:s \not = q:(v:s)\]
    \item[Identitäten] Gleichungen der form $qx=r$ $(q;r \in \mathbb{Q});\ q \not = 1$ sind nach $x$ für $x\in \mathbb{Q}$ lösbar: $x = r \cdot q^{-1}$ \\
    \begin{tabular}[t]{cc}
        $(x+y)+z = x+(x+1)$ & (xy)z=x(yz)                 \\
        $x + y = y + x$     & $xy = yx$                   \\
        $x + 0 = x$         & $x * 1 = x$                 \\
        $x + (-x) = 0$      & $xx^{-1}$ wenn $x \not = 0$ \\
        $x - y = x + (-y)$  & $x:y = xy^{-1}$
    \end{tabular}
    \[x(y+z) = xy + xz\]
    $\mathbb{Q}$ ist ein Körper.
    Die Elemente in $\mathbb{Q}$ haben eine lineare Ordnung.
    Die Zahlen liegen dicht auf dem Zahlenstrahl
\end{description}


\subsection{Reele Zahl}
$\mathbb{R}$ ist nicht abzählbar unendlich.
\begin{description}
    \item[Problem] Es gibt keine reationale Zahl $q \in \mathbb{Q}$ mit $q^2 = 2$
    \item[Annahme] Es gibt $q \in \mathbb{Q}$ mit $q^2 = 2$. Damit gibt es $a;b \in \mathbb{Z}$ mit $q = \frac{a}{b}$, $a;b \geq 1$, $\ggt(a;b) = 1$ und
    \begin{alignat*}{2}
        q^2 = \left( \frac{a}{b}\right)^2 & = 2                                                \\
        \frac{a}{b}                       & = 2                                                \\
        a^2                               & = 2b^2                                             \\
        & \Rightarrow 2 \mid a^2                             \\
        & \Rightarrow 2 \mid a                               \\
        & \Rightarrow (\exists a_0 \in \mathbb{Z})\ a = 2a_0 \\
        & \Rightarrow \left(2a_0\right)^2 = 2b^2             \\
        & \Rightarrow 4a_0^2 = 2b^2                          \\
        & \Rightarrow 2a_0^2 = b^2                           \\
        & \Rightarrow 2 \mid b^2                             \\
        & \Rightarrow 2 \mid b                               \\
    \end{alignat*}
    Aber $\ggt(a;b) = 1$
    \item[unendlicher Dezimalbruch] $d$ beteht aus 3 Dingen (Tripel)
    \begin{itemize}
        \item Vorzeichen: + oder - (bzw. +1, -1)
        \item natürlich Zahl $d_0 \in \mathbb{N}$
        \item Folge von Dezimalziffern ($f : \mathbb{N}^+ \longrightarrow \lbrace 0; 1; 2; \dots ; 9 \rbrace$)
    \end{itemize}
    Schreibweise: $d = \pm d_0,d_1 d_2 d_3\dots$\\
    $\mathbb{D} :$ Menge aller unendlichen Dezimalbrüche ist nicht $\mathbb{R}$. Lineare Ordnung $\leq$ auf $\mathbb{D}$, lexikographisch
    \item[periodischer Dezimalbruch] $d$ periodisch
    $\Leftrightarrow (\exists k \geq 0)(\exists l \geq 1) (\forall i > k)\ d_i= d_{i + l}$ $l$, also mit kleinstmöglicher Periodenlänge z.B. $5{,}72\overline{13}$
    \item[abbrechender Dezimalbruch] z.B. $102{,}53\overline{0} = 102{,}53$
    \item[unmittelbarer Nachfolger] 9er ende z.B. $2{,}1\overline{9} = 2{,}2$
    \item[Definition] Menge der Reelen Zahlen $\mathbb{R} = \lbrace \pm d | \pm d $ ist unendlicher Dezimalbruch mit zusatzvereinbarungen: $ -0 = +0 \textrm{ und } 0{,}\overline{9} = 1 \rbrace$
    \item[Rationale Zahlen in den Reelen] \
    \begin{itemize}
        \item[abbrechend] $d_0{,}d_1 d_2 \dots d_k \longmapsto d_0 + \frac{d_1}{10}+\frac{d_2}{100} + \dots + \frac{d_k}{10^k}$
        \item[beliebig] $e_0{,}e_1 e_2 \dots e_{k + 1} \dots \longmapsto$ k-te Nährung $e_0{,}e_1 e_2 \dots e_k$
    \end{itemize}
    \item[Umrechung $\mathbb{D}$ nach $\mathbb{Q}$]
    \begin{alignat*}{1}
        x         & = 3{,}1\overline{72}                                          \\
        10^{2}x     & = 317{,}2\overline{72}                                        \\
        10^{2}x-x   & = 317{,}2\overline{72} - 3{,}1\overline{72} = 317{,}2 - 3{,}1 \\
        (10^2-1)x & = 314{,}1                                                     \\
        990x      & = 3141                                                        \\
        x         & = \frac{3141}{990}                                            \\
        x         & = \frac{349}{110}                                             \\
    \end{alignat*}
    \item[Supremum und Infimum] Sei $A \subseteq \mathbb{R}; A \not = \emptyset$. $s \in \mathbb{R}$ heißt obere Schranke wenn $(\forall a \in A)\ a \leq s$ und untere schranke wenn $(\forall a \in A)\ s \leq a$. Wenn für $A$ eine obere Schranke existiert, dann heißt $A$ nach oben beschrenkt. Wenn für $A$ eine untere Schranke existiert, dann heßt $A$ nach unten beschränkt. $A$ heißt beschränkt, wenn $A$ nach oben und unten beschränkt ist. $s$ heißt Supremum von $A$, $s = \sup(A)$,wenn $s$ obere Schranke für $A$ ist und $(\forall s' \in \mathbb{R})\ s' \leq s \Rightarrow s'$ ist keine obere Schranke. $s$ heißt Infimum von $A$, $s = \infim(A)$,wenn $s$ untere Schranke für $A$ ist und $(\forall s' \in \mathbb{R})\ s' \geq s \Rightarrow s'$ ist keine untere Schranke.

    Satz: Wenn $A \subseteq \mathbb{R}; A \not = \emptyset,\ A$ nach oben beschränkt $\Rightarrow (\exists s \in \mathbb{R}) s = \supri(A)$ Analog dazu das Infimum. In $\mathbb{Q}$ gilt dass nicht.

    \item[Operationen] Bezüglich + und $\cdot$ gelten die selben Identitäten wie in $\mathbb{Q}$. $\mathbb{R}$ bilden einen Körper.
    \begin{description}
        \item[Adiition] $d + e \coloneqq \supri \lbrace d^{[k]} + e^[k] | k \in \mathbb{N}^+ \rbrace$
        \item[]
    \end{description}
    \item[normalized scientific notation] $6,674 \cdot 10^{-11}$
    \item[Intervall] \
    \begin{alignat*}{1}
        \lbrack a; b \rbrack & = \lbrace x | a \leq x\leq b \rbrace \\
        \rbrack a; b \lbrack & = \lbrace x | a < x < b \rbrace
    \end{alignat*}
    \item[erweiterte reele Zahlen] $+\infty$ und $-\infty$ (keine reelen Zahlen) $\mathbb{R}^+ = (0; \infty)$, $\mathbb{R}_0^+=\lbrack 0 ; \infty)$. In gewisser weise und ganz vosichtig kann man mit $\pm \infty$ rechnen.
    \item[irrational] $x \in \mathbb{R}; x \not \in \mathbb{Q}$ z.B.: $x = 0,10100100010000$
    \item[algebraisch] genau dann wenn, eine Nullstelle eines Polynoms mit ganzzahligen Koeffizenten.
    \item[transzendent] also nicht algebraisch $e; \pi$
\end{description}


\subsection{Additionssysteme}
``Strichliste (mit Abkürzungen)'' \\
Z.B.: $5 = ||||| = \cancel{||||}$ oder römische Ziffern: \\
\begin{tabular}[t]{|c|c|c|c|c|c|c|c|}
    \hline
    Großbuchstaben & \rom{1} & \rom{5} & \rom{10} & \rom{50} & \rom{100} & \rom{500} & \rom{1000} \\ \hline
    Wert           & 1       & 5       & 10       & 50       & 100       & 500       & 1000       \\ \hline
\end{tabular}


\subsection{Positionssysteme}
\begin{itemize}
    \item Basis $B$, $B \in \mathbb{N}$, $B>=2$
    \item Ziffern für $0$ bis $B-1$. Jede Ziffer ein Zeichen.
    \item Zahl $= \dots z_2B^2+z_1B^1+z_0B^0+z_{-1}B^{-1} \dots$
\end{itemize}
\subsubsection{Umrechnung}
\begin{description}
    \item[Polynom] $(z_{n-1}B^{n-1}z_{n-2}B^{n-2} \dots z_1B^{1}z_0B^{0})_{(B)}$
    \item[zu kleinere Basis] Fortgesetzte ganzzahlige Division mit Rest
    $217_{(10)}$ zur Basis 3 \\
    \begin{tabular}{r c}
        217 & 1 \\
        72  & 0 \\
        24  & 0 \\
        8   & 2 \\
        2   & 2 \\
        0   & 0
    \end{tabular} $217_{(10)} = 22001_{(3)}$
    \item[zu größerer Basis] mit Horner-Schema zum Dezimalsystem:

    \begin{tabular}{|c||c|c|c|c|c|} \hline
    Ziffern & 2 & 2 & 0  & 0  & 1   \\ \hline \hline
    $B=3$   & 0 & 6 & 24 & 72 & 216 \\ \hline
    & 2 & 8 & 24 & 72 & 217 \\ \hline
    \end{tabular} Addition $\downarrow$ dann Multiplikation $\nearrow$ mit $B$

    Wenn die Zielbasis eine Potenz der Ursprungsbasis ist, können $\log_{B_U}(B_Z)$ Stellen direkt zusammengefasst werden:
    $$(1000\ 0111\ 0001\ 1111)_{(2)}=(?)_{(16)}$$
    Hier können jeweils $\log_2(16)=4$ Stellen zusammengefasst werden:

    \begin{tabular}[t]{|l||c|c|c|c|} \hline
    $B = 2$  & 1000 & 0111 & 0001 & 1111 \\ \hline
    $B = 10$ & 8    & 7    & 1    & 15   \\ \hline
    $B = 16$ & 8    & 7    & 1    & F    \\ \hline
    \end{tabular}
\end{description}




    \section{Rechnen}\label{sec:rechnen}
    \subsection{Summe \& Produkt}
Summe: stilisiertes großes Sigma
$$\sum\limits_{i = n}^n f(i) = \left \lbrace \begin{array}{ll}
                                                 f(m) + f(m + 1) + \dots + f(n) & \textrm{falls } n \geq m \\
                                                 0                              & \textrm{sonst}           \\
\end{array} \right.$$
Summe aller Elemente $i$ in einer Menge $I$
$$\sum\limits_{i \in I}$$
Produkt: stilisiertes großes pi
$$\prod\limits_{i=m}^{n} f(i) = \left \lbrace \begin{array}{ll}
                                                  f(m) \cdot f(m + 1) \cdot \dots \cdot f(n) & \textrm{falls } n \geq m \\
                                                  1                                          & \textrm{sonst}           \\
\end{array} \right.$$
Produkt aller Elemente $i$ in einer Menge $I$
$$\prod\limits_{i \in I}$$
\begin{itemize}
    \item[$i$] Laufvaribale / Indexvaribale, kann umbenannt werden, vorausgesetzt die neue Bezeichnung kommt noch nicht vor.
    \item[] $$\sum\limits_{i = m}^n f(i) = \sum\limits_{j = m}^n f(j)$$
    $$\prod\limits_{i = m}^n f(i) = \prod\limits_{j = m}^n f(j)$$
    \item[$m$] Laufanfang
    \item[$n$] Laufende
    \item $i;m;n \in \mathbb{Z}$
    \item Indexverschiebeung: Laufbeginn und ende können modifiziert werden.
    $$\sum\limits_{i=m}^n f(i) = \sum\limits_{i=m+k}^{n+k} f(i-k)$$
    $$\prod\limits_{i=m}^n f(i) = \prod\limits_{i=m+k}^{n+k} f(i-k)$$
    \begin{alignat*}{1}
        1+3+4+\dots+(2n-3)+(2n-1) & = \sum\limits_{i = 1}^n (2i-1)             \\
        & = \sum\limits_{i = 3}^{n + 2} (2(i - 2)-1) \\
        & = \sum\limits_{i = 0}^{n - 1} (2(i + 1)-1) \\
    \end{alignat*}
    \item Auseinandernehmen:
    $$\sum\limits_{i=m}^n (f(i) + g(i)) = \sum\limits_{i=m}^n (f(i)) + \sum\limits_{i=m}^n (g(i))$$
    \item Ausklammern
    $$\sum\limits_{i=m}^n (a \cdot f(i)) = a \sum\limits_{i=m}^n f(i)$$
    Beispiele:
    $$\sum\limits_{i=1}^n (2i-1) = \sum\limits_{i=1}^n (2i) - \sum\limits_{i=1}^n (1) = 2\sum\limits_{i=1}^n (i) - n$$
    $$\sum\limits_{i=1}^{100} (3i-4) = 3\sum\limits_{i=1}^{100} (i) - 400$$
    \item Doppelsummen
    $$\sum\limits_{i=m}^n \sum\limits_{j=a}^b f(i; j) = \sum\limits_{j=a}^b \sum\limits_{i=m}^n f(i; j)$$
\end{itemize}
\subsection{Vereinigung \& Schnitt}
$$\bigcup\limits_{i=m}^n A(i) = \left\lbrace \begin{array}{ll}
                                                 A(m) \cup A(m + 1) \cup + \dots + \cup A(n) & \textrm{falls } m \leq n \\
                                                 \emptyset                                   & \textrm{sonst}           \\
\end{array}  \right.$$
$$\bigcap\limits_{i=m}^n A(i) = \left\lbrace \begin{array}{ll}
                                                 A(m) \cap A(m + 1) \cap + \dots + \cap A(n) & \textrm{falls } m \leq n \\
                                                 \mathbb{M}                                  & \textrm{sonst}           \\
\end{array}  \right.$$
\subsection{Potenzgestze}
\begin{itemize}
    \item $x \in \mathbb{R};\ x^0 = 1$ auch $0^0=1$
    \item $x \in \mathbb{R};\ n \in \mathbb{N}^+;\ x^n = x \cdot x \cdot x \cdot x \dots \cdot x$ n mal
    \item $x \in \mathbb{R};\ x \not = 0;\ n = -1;\ x^{-1} \coloneqq \frac{1}{x}$ $0^{-1}$ nicht definiert
    \item $x \in \mathbb{R};\ a \not = 0;\ n = -m;\ m \in \mathbb{N}^+;\ x^{-m} = \frac{1}{x^{m}} = (x^{-1})^m$
    \item $x \in \mathbb{R};\ x \geq 0;\ m \in \mathbb{N}^+;\ x^{\frac{1}{m}} \coloneqq \sqrt[m]{x}$
    \item $x \in\mathbb{R};\ x > 0;\ m \in \mathbb{N}^+;\ x^{-\frac{1}{m}} \coloneqq (x^{-1})^{\frac{1}{m}}= (\frac{1}{x})^{\frac{1}{m}}=\sqrt[m]{\frac{1}{x}} = \frac{1}{\sqrt[m]{x}}$
    \item $x \in\mathbb{R};\ x \geq 0;\ m;n \in \mathbb{N}^+;\ x^{\frac{n}{m}} \coloneqq \left(\sqrt[m]{x}\right)^n = \sqrt[m]{x^n}$
    \item $x \in \mathbb{R};\ x > 0;\ m;n \in \mathbb{N}^+;\ m^{-\frac{n}{m}} \coloneqq \frac{1}{x^{\frac{n}{m}}} = \sqrt[m]{\frac{1}{x^n}} = \left(\frac{1}{\sqrt[m]{x}}\right)^n$
    \item $x \in \mathbb{R};\ x > 0;\ (x \geq 0 \textrm{ falls } \alpha > 0);\ \alpha \in \mathbb{R};\ x^\alpha$ als Grenzwert $x^{\alpha_k} = \lim\limits_{k \to \infty}\alpha_k = \alpha;\ \alpha \in \mathbb{Q}$
    $$\exp(z) = e^z = \sum\limits_{i = 0}^\infty \frac{z^i}{i!}$$
\end{itemize}
Voraussetzung : $x \in \mathbb{R};\ x > 0;\ \alpha; \beta \in \mathbb{R}$
\begin{alignat*}{1}
    x^{\alpha + \beta}            & = x^\alpha \cdot x^\beta                                                                                                        \\
    x^{\alpha \cdot \beta}        & = \left(x^\alpha\right)^\beta = \left(x^\beta\right)	\alpha                                                                     \\
    x^{-\alpha}                   & = \frac{1}{x^{\alpha}}                                                                                                          \\
    x^0                           & = 1;\ x^1 = x;\ x^{-1}    = \frac{1}{x}                                                                                         \\
    \left(x \cdot y\right)^\alpha & = x^\alpha \cdot y^\beta                                                                                                        \\
    x^{y^z}                       & = x^{\left(y^z\right)}                                                                                                          \\
    \textrm{Wurzel}               & = \sqrt[m]{x}                                                                                                                   \\
    \sqrt[m]{x}                   & = x^{\frac{1}{m}}                                                                                                               \\
    \sqrt[m]{\sqrt[n]{x}}         & = \left(x^{\frac{1}{n}}\right)^\frac{1}{m} = x^{\frac{1}{m \cdot n}=\sqrt[m \cdot n]{x}}                                        \\
    \sqrt[m]{x^n}                 & = \left(\sqrt[m]{x}\right)^n = x^{\frac{n}{m}}                                                                                  \\
    \sqrt[1]{x}                   & = x                                                                                                                             \\
    \sqrt[m]{x} \cdot \sqrt[n]{x} & = x^{\frac{1}{m}} \cdot x^{\frac{1}{n}} = x^{\frac{1}{n}+\frac{1}{m}} = x^{\frac{m+n}{m \cdot n}} = \sqrt[m \cdot n]{x^{m + n}}
\end{alignat*}
\subsection{Fakultäten}
\begin{itemize}
    \item Fakultät $0! \coloneqq 1;\ (n+1)!=n!(n+1)$ wächst sehr schnell.
    $$(n \geq 1)\ n! = \prod\limits_{i=1}^n i$$
    \item Kombinatorische Bedeutung: Anzahl der Anordnungen von $n$ Gegenständen in einer Reihe.
    \item Näherung durch Stirling-Formel:
    $$n! \approx \sqrt{2 \pi n} \left(\frac{n}{e}\right)^n$$
    \item Nährung durch Bill Gosper
    $$n! \approx \sqrt{2 \pi n + \frac{\pi}{3}}\left(\frac{n}{e}\right)^n$$
\end{itemize}
\subsection{Binominialkoeffizient}
$n \in \mathbb{N}; m \in \mathbb{N};\ \binom{n}{k}$  gelesen ''n über m'' $n < m \Rightarrow \binom{n}{m} = 0$ $n \geq m \Rightarrow \binom{n}{0} = 1,\ \binom{n}{1} = n,\ \binom{n}{n}$
$$\binom{n}{m} = \frac{n(n-1) \dots \cdot (n-m-1)}{1\cdot 2\cdot3 \dots m} = \frac{n!}{m!(n-m)!}$$
$$\binom{n}{m}=\binom{n}{n-m}$$ Jeweils $m$ viele Faktoren, da sich der Rest wegkürzt.
z.b:
$$\binom{4}{2} = \frac{4 \cdot 3 \cdot \cancel{2 \cdot 1}}{2 \cdot 1 \cdot \cancel{2 \cdot 1}} = \frac{12}{2} = 6;\ \binom{5}{3} = \binom{5}{2} = \frac{5 \cdot 4 \cdot  \cancel{3 \cdot 2 \cdot 1}}{2 \cdot 1 \cdot \cancel{3 \cdot 2 \cdot 1}} = 10$$
Kombinatorische Bedeutung: Anzahl der m-elementigen Teilmengen einer n-elementigen Menge
$$\binom{n}{m}+\binom{n}{m+1}=\binom{n+1}{m+1}$$
\begin{alignat*}{1}
    \binom{n}{m}+\binom{n}{m+1} & =\frac{n!}{m!(n-m)!}+\frac{n!}{(m+1)!(n-m-1)!}=\frac{n!(m+1)+n!(n-m)}{(m+1)!(n-m)!} \\
    & =\frac{(n+1)!}{(m+1)!((n+1)-(m+1))!}
\end{alignat*}
Pascalsches Dreieck: \\
\begin{tabularx}{\linewidth}{@{}*2{>{\centering\arraybackslash}X}}
    \begin{tikzpicture}
        \node (11) at (0, 0) {1};
        \node (21) at (-0.5, -0.5) {1};
        \node (22) at (0.5, -0.5) {1};
        \node (31) at (-1, -1) {1};
        \node (32) at (0, -1) {2};
        \node (33) at (1, -1) {1};
        \node (41) at (-1.5, -1.5) {1};
        \node (42) at (-0.5, -1.5) {3};
        \node (43) at (0.5, -1.5) {3};
        \node (44) at (1.5, -1.5) {1};
        \node (51) at (-2, -2) {1};
        \node (52) at (-1, -2) {4};
        \node (53) at (0, -2) {6};
        \node (54) at (1, -2) {4};
        \node (55) at (2, -2) {1};
        \node (61) at (-2.5, -2.5) {1};
        \node (62) at (-1.5, -2.5) {5};
        \node (63) at (-0.5, -2.5) {10};
        \node (64) at (0.5, -2.5) {10};
        \node (65) at (1.5, -2.5) {5};
        \node (66) at (2.5, -2.5) {1};
    \end{tikzpicture} &
    \begin{tikzpicture}
        \node (11) at (0, 0) {$\binom{0}{0}$};
        \node (21) at (-0.5, -0.5) {$\binom{1}{0}$};
        \node (22) at (0.5, -0.5) {$\binom{1}{1}$};
        \node (31) at (-1, -1) {$\binom{2}{0}$};
        \node (32) at (0, -1) {$\binom{2}{1}$};
        \node (33) at (1, -1) {$\binom{2}{2}$};
        \node (41) at (-1.5, -1.5) {$\binom{3}{0}$};
        \node (42) at (-0.5, -1.5) {$\binom{3}{1}$};
        \node (43) at (0.5, -1.5) {$\binom{3}{2}$};
        \node (44) at (1.5, -1.5) {$\binom{3}{3}$};
        \node (51) at (-2, -2) {$\binom{4}{0}$};
        \node (52) at (-1, -2) {$\binom{4}{1}$};
        \node (53) at (0, -2) {$\binom{4}{2}$};
        \node (54) at (1, -2) {$\binom{4}{3}$};
        \node (55) at (2, -2) {$\binom{4}{5}$};
        \node (61) at (-2.5, -2.5) {$\binom{5}{0}$};
        \node (62) at (-1.5, -2.5) {$\binom{5}{1}$};
        \node (63) at (-0.5, -2.5) {$\binom{5}{2}$};
        \node (64) at (0.5, -2.5) {$\binom{5}{3}$};
        \node (65) at (1.5, -2.5) {$\binom{5}{4}$};
        \node (66) at (2.5, -2.5) {$\binom{5}{5}$};
    \end{tikzpicture}
\end{tabularx}
$$\sum\limits_{k=1}^n \binom{k}{m} = \binom{1}{m}+\binom{2}{m}+ \dots + \binom{m}{n}+\binom{m}{m}+\binom{m+1}{m}+\dots+\binom{n}{m}=\binom{n+1}{m+1}$$
$$\sum\limits_{k=1}^n\binom{k}{1} = \frac{n(n+1)}{2}$$
Binomialsatz
$$(a+n)^n=\sum\limits_{m=0}^n\binom{n}{m}a^{n-m}b^m=a^n+\binom{n}{1}a^{n-1}b+\binom{n}{2}a^{n-2}b^2+\dots + \binom{n}{n-1}ab^{n-1}+b^n$$
\subsection{Umformungen von Termen}
Erklärung: Ein (Funktions)Term ist ein ''vernünftig'' aufgebauter Ausdruck zur Berechnung einer Funktion.

Terme könne aus folgendem bestehen
\begin{itemize}
    \item Zeichen für Variablen und Parameter $x;\ y;\ z;\ a;\ b$
    \item Zahlen, Konstanten
    \item Operationen
    \item Funktionszeichen $\exp;\ \sin;\ cos;\ $
    \item technische Zeichen $(;);\lbrace;\rbrace,\lbrack,\rbrack$
\end{itemize}
Funktionsbezeichung: $f;\ f(x);\ f(x;y)$ Wir bezeichnunen Terme ähnlich wie Funktionen. Aber: Ein Term definiert eine Funktion aber nicht umgekehrt.


\underline{Ziel:} Möglichst einfache Terme für eine Funktion finden. Zu einem Term $f(x)$ gehört ein maximaler Definitionsbereich (auch natürlicher Definitionsberiech). Das ist die größte Teilmenge $D \in \mathbb{R}$ für die alle Teilterme von $f$ definiert sind. Dieser DB kann eventuell weiter eingeschränkt werden. Bezeichnungen für den Definitionsberiech: $D_f;\ \textrm{DBb}(f), D$


Beispiel: $f(x) = \frac{x^2-x-6}{x+2}\quad D_f=\mathbb{R}\setminus\lbrace-2\rbrace$
$$f(x)= \frac{(x-3)(x+2)}{x+2} = x-3\quad | x \not = -2$$

\subsubsection{Faktorisieren}
\begin{description}
    \item[Binomische Formeln] \
    \begin{itemize}
        \item[1.] $(a+b)^2=a^2+2ab+b^2$
        \item[2.] $(a-b)^2=a^2-2ab+b^2$
        \item[3.] $(a-b)(a+b)=a^2-b^2$
    \end{itemize}
    \item[Summenformel] \
    \begin{itemize}
        \item $(1-x)(1+x+x^2+\dots +x^n)=1-x^{n+1}$
        \item $(x-1)(1+x+x^2+\dots +x^n)=x^{n+1}-1$
    \end{itemize}
    \item[Distributivgesetze] \
    \begin{itemize}
        \item $a(a+b)=ab+ac$
        \item $(a+b)(c+d) = ac +ad + bc +bd$
    \end{itemize}
    \item[Vieta] $(x-a)(x-b) = x^2 - (a+b)x+ab$
    \item[Wurzel aus Nenner] $\frac{1}{\sqrt{2}}=\frac{\sqrt{2}}{2}$
    $$\frac{2+\sqrt{3}}{2-\sqrt{3}}=\frac{2+\sqrt{3}}{2-\sqrt{3}} \cdot \frac{2+\sqrt{3}}{2+\sqrt{3}}=\frac{4+4\sqrt{3}+3}{4-3}=7+4\sqrt{3}$$
\end{description}
\subsection{Proportionalität}
Größe
\begin{itemize}
    \item Bezeichnung $X$, z.B. Fahrstrecke
    \item zugehörige Wertemenge, hier stehts $X\in \mathbb{R}$ (notfalls runden mit Verstand)
    \item eventuell mit Einheit, schreibweise $x \in X$
    \item Zwei Größen $x;\ Y$ i.a. nicht unabhänig.
    \item $E \subseteq X \times Y$
    \item $X$ und $Y$ heißen proportional, $X \sim Y$, wenn $(\exists c)\ (x;y) \in E \Leftrightarrow \frac{x}{y}=c$ z.B. Farstrecke $\sim$ Benzinverbrauch
    \item $X$ und $Y$ heißen umgekehrt proportional, $x \sim \frac{1}{Y}$, wenn $(\exists c)\ (x;y) \in E \Leftrightarrow x \cdot y = c$. Z.B.: Arbeiteranzahl $\sim \frac{1}{\textrm{Arbeitszeit}}$
    \item $X \sim Y;\ (x_1;y_1);\ (x_2;y_2) \in E \Rightarrow \frac{x_1}{y_1} = c = \frac{x_2}{y_2}$
    \item $X \sim \frac{1}{Y};\ (x_1;y_1);\ (x_2;y_2) \in E \Rightarrow x_1 \cdot y_1 = c = x_2 \cdot y_2$
    \item mehr als zwei Größen: Für die Zerlegung von 7,2t brauchen 14 Arbeiter 8h. Wie viele Arbeiter braucht man, um 6t in 8 h zu zerlgen. \\
    Propotionalitätsbeziehung zwichen $$E \subseteq X \times  Y \times Z\ : (x; y; z) \in E$$ $$(\exists i_x; i_y; i_z \in \lbrace -1; 1 \rbrace) (\exists c)\ (x;y;z) \in E \Leftrightarrow x^{i_x}y^{i_y}z^{i_z} = c$$
    $$A \sim S;\ A \sim \frac{1}{T}$$
    $$\frac{14z \cdot 8h}{7{,}2t}=c=\frac{a \cdot 8h}{6t} \Rightarrow a = \frac{6t}{7{,}2t}\cdot \frac{8h}{8h} \cdot 14z \approx 12z$$
\end{itemize}
\subsubsection{Prozentrechnung}
\begin{itemize}
    \item Spezialfall der Proportionalität
    \item Zwei größen:\
    \begin{itemize}
        \item Prozente
        \item andere Größe
    \end{itemize}
    \item $1\% = \frac{1}{100}$
    \item $1\permil = \frac{1}{1000}$
    \item Grundwert $G \hat{=} 100\%$, Prozentwert $W \hat{=} p\%$
    $$\frac{G}{100\%} = \frac{W}{p\%}$$
\end{itemize}
\subsection{Gleichungen}
$$f(x)=g(x)$$
\begin{description}
    \item[Lösungsmenge] $\mathbb{L} = \lbrace x \in D_f \cap D_g | f(x) = g(x) \rbrace$ explizit angeben.
    \item[Äquivalenete Umformung] ändert $\mathbb{L}$ nicht. Zwei gleichungen heißen äquivalent wenn ihre Lö-sungsmengen gleich sind.
    $$f(x)= g(x) \Leftrightarrow \widetilde{f}(x) = \widetilde{g}(x)$$
    \item[nichtäquivalente Umformungen]
    Folgerungen $f(x) = g(x) \Rightarrow \widetilde{f}(x) = \widetilde{g}(x)$, d.h. $\mathbb{L} \subseteq \widetilde{\mathbb{L}}$ Probe!
    \begin{alignat*}{2}
        x-2     & = 3 \quad    &  & \mathbb{L} = \lbrace 5 \rbrace     \\
        (x-2)^2 & = 3^2  \quad &  & \mathbb{L} = \lbrace -1; 5 \rbrace \\
    \end{alignat*}
    \item[spezielle Umformungen] $t(x)$ sei ein weiterer Term mit $D_f \cap D_g \subseteq D_f$
    \begin{alignat*}{2}
        f(x) = g(x) & \Leftrightarrow f(x) + t(x)       &  & = g(x) + t(x)                                     \\
        f(x) = g(x) & \Leftrightarrow f(x) \cdot t(x)   &  & = g(x) \cdot t(x)\textrm{ falls } t(x) \not = 0   \\
        f(x) = g(x) & \Leftrightarrow \frac{f(x)}{t(x)} &  & = \frac{g(x)}{t(x)}\textrm{ falls } t(x) \not = 0 \\
    \end{alignat*}
    $h : D_h \longrightarrow \mathbb{R}$ Funktion, , $f\lbrack D_f \cap D_g \rbrack \cup g\lbrack D_f \cap D_g \rbrack \subseteq D_h$
    $$f(x) = g(x) \Rightarrow h(f(x)) = h(g(x))$$
    Ist $h$ insbesondere umkehrbar (injektiv), dann gilt
    $$f(x) = g(x) \Leftrightarrow h(f(x)) = h(g(x))$$
    \item[Lineare Gleichungen] $ax + b = 0 \quad a \not = 0$
    \begin{alignat*}{2}
        & \Leftrightarrow ax &  & = -b           \\
        & \Leftrightarrow x  &  & = \frac{-b}{a} \\
    \end{alignat*}
    \begin{alignat*}{3}
        & \quad \frac{7x+91}{17x+221} &  & = 11                 &  & | \cdot (17x + 221) \\
        & \Leftrightarrow 7x + 91     &  & = 11 \cdot (17x+221)                          \\
        & \Leftrightarrow 7x + 91     &  & = 187x + 2431                                 \\
        & \Leftrightarrow 0           &  & = 180x +2340         &  & |-2340              \\
        & \Leftrightarrow 180x        &  & = -2340              &  & | :180              \\
        & \Leftrightarrow x           &  & = -13                                         \\
    \end{alignat*}
    $$\mathbb{D} = 17x+221 \not = 0$$
    $$x \not \in \mathbb{D}$$
    \item[Gleichung mit Beträgen] $|a| = \sqrt{a^2}$
    $$\begin{array}{ll}
          |a| \geq 0                  & |a| = 0 \Leftrightarrow a = 0              \\
          |a \cdot b| = |a| \cdot |b| & \frac{|a|}{|b|} = \left|\frac{a}{b}\right| \\
          |a+b| \leq  |a| + |b|       & ||a|-|b|| \leq |a-b|                       \\
    \end{array}$$
    \begin{itemize}
        \item $|f(x)| = c$ \
        \begin{itemize}
            \item falls $c<0$, so $\mathbb{L}= \emptyset$
            \item falls $c\geq0$: $|f(x)| = c \Leftrightarrow f(x) = c \vee f(x) = -c;\ \mathbb{L} = \mathbb{L}_1 \cup \mathbb{L}_2$
        \end{itemize}
        \item $|f(x)| = |g(x)| \Leftrightarrow f(x)^2 = g(x)^2$
        \item $|f(x)| = |g(x)| \Leftrightarrow \left|\frac{f(x)}{g(x)}\right| = 1 \Leftrightarrow \frac{f(x)}{g(x)} = 1 \vee \frac{f(x)}{g(x)} = -1$
        \item[Allgemein:] Vollständige Fallunterschiedung
        $$|x-1|+|x+1|=10$$
        \begin{itemize}
            \item[Fall 1] $x-1 \geq 0;\ x+1 \geq 0 \ \Leftrightarrow x \geq 1;\ x\geq -1 \Leftrightarrow x>1$
            \begin{alignat*} {1}
            (x-1) + (x+1) & = 10                \\
            2x            & = 10                \\
            x             & = 5                 \\
            \mathbb{L}_1  & = \lbrace 5 \rbrace \\
            \end{alignat*}
            \item[Fall 2] $x-1 \geq 0;\ x+1 < 0 \ \Leftrightarrow x \geq 1;\ x < -1 \quad \mathbb{L}_2 = \emptyset$
            \item[Fall 3] $x-1 < 0;\ x+1 \geq 0 \ \Leftrightarrow x < 1;\ x\geq -1 \Leftrightarrow -1 \leq x < 1$
            \begin{alignat*} {1}
                -(x-1) + (x+1) & = 10        \\
                2              & = 10        \\
                \mathbb{L}_3   & = \emptyset \\
            \end{alignat*}
            \item[Fall 4] $x-1 < 0;\ x+1 < 0 \ \Leftrightarrow x < 1;\ x < -1 \Leftrightarrow x<-1$
            \begin{alignat*} {1}
                -(x-1) +(-(x+1)) & = 10                 \\
                -2x              & = 10                 \\
                x                & = -5                 \\
                \mathbb{L}_1     & = \lbrace -5 \rbrace \\
            \end{alignat*}
            \item[$\mathbb{L}$] = $\mathbb{L}_1 \cup \mathbb{L}_2 \cup \mathbb{L}_3 \cup \mathbb{L}_4 = \lbrace -5; 5\rbrace$
        \end{itemize}
    \end{itemize}
    \item[Quadratische Funktionen]
\end{description}


\end{document}
