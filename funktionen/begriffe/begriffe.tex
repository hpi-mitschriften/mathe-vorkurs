\begin{description}
    \item[Definition] Zur Definition einer Funktion $f$ braucht man drei Dinge
    \begin{itemize}
        \item Menge $A$, der Definitionsbereich von $f$, $A = D_f$
        \item Menge $B$, der Wertevorrat von $f$, $B = W_f$
        \item Eine Zuordnung, die jedem $a \in A$ genau ein Element $b \in B$ zuordnet \\
        Schreibweise: $b = f(a)$ bzw. $a \longmapsto f(a)$ \\
        Mathematisch wird diese Zuordnung gegeben durch eine Menge von geordneten Paaren
        \[\textrm{Graph}(f) = \lbrace(a,f(a)) | a \in A \rbrace \subseteq A \times B\]
        mit den Eigenschaften:
        \begin{itemize}
            \item $(\forall a \in A)(\exists b \in B)\ (a;b) \in \textrm{Graph}(f)$ (Vollständigkeit)
            \item $(\forall a \in A)(\forall b_1,b_2 \in B)\ (a;b_1);(a;b_2) \in \textrm{Graph}(f)\Rightarrow b_1 = b_2$ (Eindeutigkeit)
        \end{itemize}
    \end{itemize}
    \item[Schreibweise]
    \begin{alignat*}{3}
        f :\  & A \longrightarrow &  & B \quad , \quad a &  & \longmapsto f(a) = \cdots \\
        & D_f               &  & W_v               &  & \textrm{Graph}
    \end{alignat*}
    \item[Bild] Die Menge aller Funktionswerte von $f$. $\lbrace f(a) | a \in A \rbrace = \lbrace b \in B  | (\exists a \in A) b = f(a) \rbrace \subseteq B$
    \item[surjektiv] $(\forall b \in B)(\exists a \in A) \ f(a) = b$ \\
    \begin{tabularx}{\linewidth}{l|X}
        \adjustbox{valign = t}{
            \begin{tikzpicture}[thick, set/.style = {ellipse, minimum width = 2cm, minimum height = 4cm, draw = black, align = center}, element/.style = {circle, draw = black, minimum size = 0.7, outer sep = 0.05cm}]
                \node [set, label={90:$A$}] (A) at (-1.5,0) {};
                \node [set, label={90:$B$}] (B) at (1.5,0) {};
                \node [element] (1) at (-1.5, 1.5) {1};
                \node [element] (2) at (-1.5, 0.5) {2};
                \node [element] (3) at (-1.5, -0.5) {3};
                \node [element] (4) at (-1.5, -1.5) {4};
                \node [element] (A) at (1.5, 1.5) {A};
                \node [element] (B) at (1.5, 0.5) {B};
                \node [element] (C) at (1.5, -0.5) {C};
                \draw [->] (1) to (A);
                \draw [->] (2) to (B);
                \draw [->] (3) to (C);
                \draw [->] (4) to (C);
            \end{tikzpicture}
        } &
        Für jedes Element in $B$ existiert (mindestens) ein Urbild in $A$.
        Für jede rein surjektive Abbildung gilt:
        \[|A|>|B|\] \\ \hline
    \end{tabularx}
    \item[injektiv] $(\forall a_1,a_2 \in A) (a_1 \not = a_2 \Rightarrow f(a_1) \not = f(a_2))$ \\
    \begin{tabularx}{\linewidth}{l|X}
        \adjustbox{valign = t}{
            \begin{tikzpicture}[thick, set/.style = {ellipse, minimum width = 2cm, minimum height = 4cm, draw = black, align = center}, element/.style = {circle, draw = black, minimum size = 0.7, outer sep = 0.05cm}]
                \node [set, label={90:$A$}] (A) at (-1.5,0) {};
                \node [set, label={90:$B$}] (B) at (1.5,0) {};
                \node [element] (1) at (-1.5, 1.5) {1};
                \node [element] (2) at (-1.5, 0.5) {2};
                \node [element] (3) at (-1.5, -0.5) {3};
                \node [element] (A) at (1.5, 1.5) {A};
                \node [element] (B) at (1.5, 0.5) {B};
                \node [element] (C) at (1.5, -0.5) {C};
                \node [element] (D) at (1.5, -1.5) {D};
                \draw [->] (1) to (A);
                \draw [->] (2) to (B);
                \draw [->] (3) to (D);
            \end{tikzpicture}
        } &
        Für jede zwei Elemente in $A$ gilt, dass wenn sie verschieden von einander sind, dann auch ihre Funktionswerte von $f$ verschieden sind.
        Also hat jedes Element in $B$ höchstens ein Urbild.
        Für jede rein injektive Abbildung gilt:
        \[|A|<|B|\] \\ \hline
    \end{tabularx}
    \item[bijektiv]  surjektiv $\wedge$ injektiv: $(\forall b \in B)(\exists ! a \in A)\ f(a) = b$ \\
    \begin{tabularx}{\linewidth}{l|X}
        \adjustbox{valign = t}{
            \begin{tikzpicture}[thick, set/.style = {ellipse, minimum width = 2cm, minimum height = 4cm, draw = black, align = center}, element/.style = {circle, draw = black, minimum size = 0.7, outer sep = 0.05cm}]
                \node [set, label={90:$A$}] (A) at (-1.5,0) {};
                \node [set, label={90:$B$}] (B) at (1.5,0) {};
                \node [element] (1) at (-1.5, 1.5) {1};
                \node [element] (2) at (-1.5, 0.5) {2};
                \node [element] (3) at (-1.5, -0.5) {3};
                \node [element] (4) at (-1.5, -1.5) {4};
                \node [element] (A) at (1.5, 1.5) {A};
                \node [element] (B) at (1.5, 0.5) {B};
                \node [element] (C) at (1.5, -0.5) {C};
                \node [element] (D) at (1.5, -1.5) {D};
                \draw [->] (1) to (A);
                \draw [->] (2) to (B);
                \draw [->] (3) to (C);
                \draw [->] (4) to (D);
            \end{tikzpicture}
        } &
        Für jedes Element in $B$ existiert genau ein Urbild in $A$.
        Für jede bijektive Abbildung gilt:
        \[|A|=|B|\] \\ \hline
    \end{tabularx}
    \item[Identitätsfunktion] $id_A : A \longrightarrow A , a \longmapsto a$ z.B. $f(x) = x$
    \item[Komposition] $f : A \longrightarrow B;\ g : B \longrightarrow C$
    \[(g \circ f) : A \longrightarrow C, a \longmapsto g(f(a))\]
    \begin{alignat*}{3}
        f : A \longrightarrow B \Rightarrow & f    &  & = f \circ id_A &  & = id_A \circ f \\
        & f(a) &  & = f(id_A(a))   &  & = id_A(f(a))
    \end{alignat*}
\end{description}
