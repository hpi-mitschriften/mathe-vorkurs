\begin{description}
    \item[Aussageform] Haben die Form einer Aussage, enthalte aber Variablen.
    \[3 + x = 5;\ A(x);\ B(x;y)\]
    \begin{itemize}
        \item werden zu Aussagen, wenn die Variablen belegt werden.
        Für die Variablen ist ein eingrenzender Grundbereich vorzugeben.
        Z.~B.: $x \in \mathbb{N}$
        \item Wie Aussagen kann man Aussageformen miteinander Verknüpfen (mit Junktoren) und man erhält neue Aussageformen.
    \end{itemize}
    \item[Quantoren] Außer der Belegung der Variablen mit Werten, gibt es noch andere Möglichkeiten aus einer Aussageform eine Aussage zu machen.
    Ein Grundbereich $M$ muss vorgegeben sein.

    ``Für alle x aus $M$ gilt A(x)'' \newline
    Für alle $x \in \mathbb{N}$ gilt $3 + x = 5$ (falsche Aussage) kurz mit Allquantor $\forall$ :
    \[(\forall x \in \mathbb{N})\ 3 + x = 5\]

    ``Es existiert ein x aus $M$ mit A(x)'' \newline
    Es existiert (mindestens) ein $x \in \mathbb{N}$ mit $3 + x = 5$ (wahre Aussage) kurz mit Existenzquantor $\exists$ :
    \[(\exists x \in M)\ 3 + x = 5\]

    ``Es existiert höchsten ein x aus $M$ mit A(x)''
    \[(\forall x)(\forall y)\ (A(x) \wedge A(y) \Rightarrow x = y)\]

    ``Es existiert genau ein x aus $M$ mit A(x)''
    \[(\exists ! x) A(x) \equiv ((\exists x) A(x)) \wedge ((\forall x)(\forall y)\ (A(x) \wedge A(y) \Rightarrow x = y))\]
\end{description}
