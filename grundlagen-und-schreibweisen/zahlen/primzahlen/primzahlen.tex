\begin{description}
    \item[Teiler] $a;b \in \mathbb{Z}$\\
    $a$ ist ein Teiler von $b$, geschrieben $a \mid c$, falls $(\exists c \in \mathbb{Z})\quad a \cdot c = b$\\
    Jede ganze Zahl $b$ ist teilbar durch: 1, -1, b , -b.
    Diese heißen die trivialen Teiler von $b$.
    Eigenschaften: \\
    $a \mid 0;\, a \mid 0$ \\
    $a \mid b \wedge b \mid c \Rightarrow a \mid b$ \\
    $a \mid b \Rightarrow a \mid (-b), (-a) \mid b, (-a) \mid (-b)$ \\
    $a;b \geq 1 \wedge a \mid b \Rightarrow a \leq b$
    \item[Primzahl] Eigenschaften:
    \begin{itemize}
        \item Eine ganze Zahl $p \in \mathbb{Z}$ heißt Primzahl, wenn $p \geq 2$ und $p$ nur triviale Teiler hat.
        \item Jede ganze Zahl $b \geq 2$ hat mindesten einen Primitiver.
        \item Es gibt unendlich viele Primzahlen.
        Beweis durch Widerspruch
        \[|\mathbb{P}| \in \mathbb{N}\]
        $n$ sei die Anzahl aller Primzahl, und alle Primzahlen seien in der Menge $\mathbb{P}$.
        Man bilde $b = \prod\limits_{p \in \mathbb{ P}} + 1$.
        Dann ist $b \geq 2$ und laut Hilfssatz hat $b$ einen Primteiler, dieser sei $q$.
        Damit hat man eine Primzahl $q \not \in \mathbb{P}$ gefunden.
        Daraus folgt, dass die Konstruktion $\mathbb{P} = \lbrace p_1; \dots; p_n \rbrace | n \in \mathbb{N}$ nicht alle Primzahlen enthalten kann.
        \item Der kleinste Teiler einer Zahl $b \in \mathbb{N}|b \geq 2$ ist eine Primzahl.
        \item Der kleinste Primteiler $p$ einer Zahl $a \in \mathbb{Z};\ a \geq 2;\ a \not \in \mathbb{P}$ ist $p \leq \sqrt{a}$
    \end{itemize}
    \item[Fundamentalsatz der Arithmetik] Jede Zahl $b \geq 2$ lässt sich als Produktion von Primzahlen darstellen (Primfaktorisierung).
    Vorkommende Primzahlen und ihre Anzahl sind bis auf Reihenfolge eindeutig bestimmt.
\end{description}
