Größe
\begin{itemize}
    \item Bezeichnung $X$, z.B. Fahrstrecke
    \item zugehörige Wertemenge, hier stehts $X\in \mathbb{R}$ (notfalls runden mit Verstand)
    \item eventuell mit Einheit, schreibweise $x \in X$
    \item Zwei Größen $x;\ Y$ i.a. nicht unabhänig.
    \item $E \subseteq X \times Y$
    \item $X$ und $Y$ heißen proportional, $X \sim Y$, wenn $(\exists c)\ (x;y) \in E \Leftrightarrow \frac{x}{y}=c$ z.B. Farstrecke $\sim$ Benzinverbrauch
    \item $X$ und $Y$ heißen umgekehrt proportional, $x \sim \frac{1}{Y}$, wenn $(\exists c)\ (x;y) \in E \Leftrightarrow x \cdot y = c$. Z.B.: Arbeiteranzahl $\sim \frac{1}{\textrm{Arbeitszeit}}$
    \item $X \sim Y;\ (x_1;y_1);\ (x_2;y_2) \in E \Rightarrow \frac{x_1}{y_1} = c = \frac{x_2}{y_2}$
    \item $X \sim \frac{1}{Y};\ (x_1;y_1);\ (x_2;y_2) \in E \Rightarrow x_1 \cdot y_1 = c = x_2 \cdot y_2$
    \item mehr als zwei Größen: Für die Zerlegung von 7,2t brauchen 14 Arbeiter 8h. Wie viele Arbeiter braucht man, um 6t in 8 h zu zerlgen. \\
    Propotionalitätsbeziehung zwichen $$E \subseteq X \times  Y \times Z\ : (x; y; z) \in E$$ $$(\exists i_x; i_y; i_z \in \lbrace -1; 1 \rbrace) (\exists c)\ (x;y;z) \in E \Leftrightarrow x^{i_x}y^{i_y}z^{i_z} = c$$
    $$A \sim S;\ A \sim \frac{1}{T}$$
    $$\frac{14z \cdot 8h}{7{,}2t}=c=\frac{a \cdot 8h}{6t} \Rightarrow a = \frac{6t}{7{,}2t}\cdot \frac{8h}{8h} \cdot 14z \approx 12z$$
\end{itemize}

\subsubsection{Prozentrechnung}
\begin{itemize}
    \item Spezialfall der Proportionalität
    \item Zwei größen:\
    \begin{itemize}
        \item Prozente
        \item andere Größe
    \end{itemize}
    \item $1\% = \frac{1}{100}$
    \item $1\permil = \frac{1}{1000}$
    \item Grundwert $G \hat{=} 100\%$, Prozentwert $W \hat{=} p\%$
    $$\frac{G}{100\%} = \frac{W}{p\%}$$
\end{itemize}
