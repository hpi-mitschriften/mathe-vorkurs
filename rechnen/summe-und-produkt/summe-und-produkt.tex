Summe: stilisiertes großes Sigma
$$\sum\limits_{i = n}^n f(i) = \left \lbrace \begin{array}{ll}
                                                 f(m) + f(m + 1) + \dots + f(n) & \textrm{falls } n \geq m \\
                                                 0                              & \textrm{sonst}           \\
\end{array} \right.$$
Summe aller Elemente $i$ in einer Menge $I$
$$\sum\limits_{i \in I}$$
Produkt: stilisiertes großes pi
$$\prod\limits_{i=m}^{n} f(i) = \left \lbrace \begin{array}{ll}
                                                  f(m) \cdot f(m + 1) \cdot \dots \cdot f(n) & \textrm{falls } n \geq m \\
                                                  1                                          & \textrm{sonst}           \\
\end{array} \right.$$
Produkt aller Elemente $i$ in einer Menge $I$
$$\prod\limits_{i \in I}$$
\begin{itemize}
    \item[$i$] Laufvaribale / Indexvaribale, kann umbenannt werden, vorausgesetzt die neue Bezeichnung kommt noch nicht vor.
    \item[] $$\sum\limits_{i = m}^n f(i) = \sum\limits_{j = m}^n f(j)$$
    $$\prod\limits_{i = m}^n f(i) = \prod\limits_{j = m}^n f(j)$$
    \item[$m$] Laufanfang
    \item[$n$] Laufende
    \item $i;m;n \in \mathbb{Z}$
    \item Indexverschiebeung: Laufbeginn und ende können modifiziert werden.
    $$\sum\limits_{i=m}^n f(i) = \sum\limits_{i=m+k}^{n+k} f(i-k)$$
    $$\prod\limits_{i=m}^n f(i) = \prod\limits_{i=m+k}^{n+k} f(i-k)$$
    \begin{alignat*}{1}
        1+3+4+\dots+(2n-3)+(2n-1) & = \sum\limits_{i = 1}^n (2i-1)             \\
        & = \sum\limits_{i = 3}^{n + 2} (2(i - 2)-1) \\
        & = \sum\limits_{i = 0}^{n - 1} (2(i + 1)-1) \\
    \end{alignat*}
    \item Auseinandernehmen:
    $$\sum\limits_{i=m}^n (f(i) + g(i)) = \sum\limits_{i=m}^n (f(i)) + \sum\limits_{i=m}^n (g(i))$$
    \item Ausklammern
    $$\sum\limits_{i=m}^n (a \cdot f(i)) = a \sum\limits_{i=m}^n f(i)$$
    Beispiele:
    $$\sum\limits_{i=1}^n (2i-1) = \sum\limits_{i=1}^n (2i) - \sum\limits_{i=1}^n (1) = 2\sum\limits_{i=1}^n (i) - n$$
    $$\sum\limits_{i=1}^{100} (3i-4) = 3\sum\limits_{i=1}^{100} (i) - 400$$
    \item Doppelsummen
    $$\sum\limits_{i=m}^n \sum\limits_{j=a}^b f(i; j) = \sum\limits_{j=a}^b \sum\limits_{i=m}^n f(i; j)$$
\end{itemize}