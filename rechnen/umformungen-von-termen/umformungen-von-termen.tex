Erklärung: Ein (Funktions-)Term ist ein ``vernünftig'' aufgebauter Ausdruck zur Berechnung einer Funktion.

Terme könne aus folgendem bestehen
\begin{itemize}
    \item Zeichen für Variablen und Parameter $x;\ y;\ z;\ a;\ b$
    \item Zahlen, Konstanten
    \item Operationen
    \item Funktionszeichen $\exp;\ \sin;\ cos;\ $
    \item technische Zeichen $(;);\lbrace;\rbrace,\lbrack,\rbrack$
\end{itemize}
Funktionsbezeichnung: $f;\ f(x);\ f(x;y)$ Wir bezeichnen Terme ähnlich wie Funktionen.
Aber: Ein Term definiert eine Funktion aber nicht umgekehrt.


\underline{Ziel:} Möglichst einfache Terme für eine Funktion finden.
Zu einem Term $f(x)$ gehört ein maximaler Definitionsbereich (auch natürlicher Definitionsbereich).
Das ist die größte Teilmenge $D \in \mathbb{R}$, für die alle Teilterme von $f$ definiert sind.
Dieser DB kann eventuell weiter eingeschränkt werden.
Bezeichnungen für den Definitionsbereich: $D_f;\ \textrm{DBb}(f), D$


Beispiel: $f(x) = \frac{x^2-x-6}{x+2}\quad D_f=\mathbb{R}\setminus\lbrace-2\rbrace$
\[f(x)= \frac{(x-3)(x+2)}{x+2} = x-3\quad | x \not = -2\]

\subsubsection{Faktorisieren}
\begin{description}
    \item[Binomische Formeln] \
    \begin{itemize}
        \item[1.] $(a+b)^2=a^2+2ab+b^2$
        \item[2.] $(a-b)^2=a^2-2ab+b^2$
        \item[3.] $(a-b)(a+b)=a^2-b^2$
    \end{itemize}
    \item[Summenformel] \
    \begin{itemize}
        \item $(1-x)(1+x+x^2+\dots +x^n)=1-x^{n+1}$
        \item $(x-1)(1+x+x^2+\dots +x^n)=x^{n+1}-1$
    \end{itemize}
    \item[Distributivgesetze] \
    \begin{itemize}
        \item $a(a+b)=ab+ac$
        \item $(a+b)(c+d) = ac +ad + bc +bd$
    \end{itemize}
    \item[Vieta] $(x-a)(x-b) = x^2 - (a+b)x+ab$
    \item[Wurzel aus Nenner] $\frac{1}{\sqrt{2}}=\frac{\sqrt{2}}{2}$
    $$\frac{2+\sqrt{3}}{2-\sqrt{3}}=\frac{2+\sqrt{3}}{2-\sqrt{3}} \cdot \frac{2+\sqrt{3}}{2+\sqrt{3}}=\frac{4+4\sqrt{3}+3}{4-3}=7+4\sqrt{3}$$
\end{description}
