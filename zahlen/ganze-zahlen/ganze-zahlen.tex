\begin{description}
    \item[Motivation] $\mathbb{Z} \coloneqq \lbrace 0, 1, -1, 2, -2 \dots \rbrace$ (abzählbar)\\
    \begin{tabular}[t]{ll}
        $x+1=0$ & ist nicht lösbar in $\mathbb{N}$                               \\
        $x+a=0$ & man nimmt zu jeder Zahl $a \in \mathbb{N}$ eine Gegenzahl $-a$
    \end{tabular} \\
    Lösung für $x+a=0$ (Ausnahme: $a=0$, denn $-0=0$)
    \item[Operationen] $+; \ -; \ \cdot$
    \item[spezielle Elemente] $0,\ 1$
    \item[lineare Ordnung] $<;\ \leq;\ >;\ \geq$
    \item[Gesetze] $(\forall a \in \mathbb{Z})$ gilt: \\
    \begin{tabular}{l|c|c}
        & Addition            & Multiplikation                              \\ \hline
        & $a+0 = a$           & $a \cdot 1 = a$                             \\ \hline
        Kommutativ & $a+b = b+a$         & $a \cdot b = b \cdot a $                    \\ \hline
        Assoziativ & $(a+b)+c = a+(b+c)$ & $(a \cdot b) \cdot c = a \cdot (b \cdot c)$ \\ \hline
        & $a+(-a) = 0$
    \end{tabular} \\
    Ring-Identitäten: $a \cdot (b + c) = a \cdot b + a \cdot c$
    \item[Betrag] $|a| = \left\lbrace \begin{array}{rc} a & a \geq 0 \\ -a & a < 0\end{array} \right.$
    \item[Division] Es sein $a;m \in \mathbb{Z}|m \geq 1$ dann gibt es $q \in \mathbb{Z}$ mit $a=q \cdot m + r$ und $0 \leq r < m$. $q;r$ sind eindeutig bestimmt
\end{description}