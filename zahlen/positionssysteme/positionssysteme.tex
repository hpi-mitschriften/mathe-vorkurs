\begin{itemize}
    \item Basis $B$, $B \in \mathbb{N}$, $B>=2$
    \item Ziffern für $0$ bis $B-1$. Jede Ziffer ein Zeichen.
    \item Zahl $= \dots z_2B^2+z_1B^1+z_0B^0+z_{-1}B^{-1} \dots$
\end{itemize}
\subsubsection{Umrechnung}
\begin{description}
    \item[Polynom] $(z_{n-1}B^{n-1}z_{n-2}B^{n-2} \dots z_1B^{1}z_0B^{0})_{(B)}$
    \item[zu kleinere Basis] Fortgesetzte ganzzahlige Division mit Rest
    $217_{(10)}$ zur Basis 3 \\
    \begin{tabular}{r c}
        217 & 1 \\
        72  & 0 \\
        24  & 0 \\
        8   & 2 \\
        2   & 2 \\
        0   & 0
    \end{tabular} $217_{(10)} = 22001_{(3)}$
    \item[zu größerer Basis] mit Horner-Schema zum Dezimalsystem:

    \begin{tabular}{|c||c|c|c|c|c|} \hline
    Ziffern & 2 & 2 & 0  & 0  & 1   \\ \hline \hline
    $B=3$   & 0 & 6 & 24 & 72 & 216 \\ \hline
    & 2 & 8 & 24 & 72 & 217 \\ \hline
    \end{tabular} Addition $\downarrow$ dann Multiplikation $\nearrow$ mit $B$

    Wenn die Zielbasis eine Potenz der Ursprungsbasis ist, können $\log_{B_U}(B_Z)$ Stellen direkt zusammengefasst werden:
    $$(1000\ 0111\ 0001\ 1111)_{(2)}=(?)_{(16)}$$
    Hier können jeweils $\log_2(16)=4$ Stellen zusammengefasst werden:

    \begin{tabular}[t]{|l||c|c|c|c|} \hline
    $B = 2$  & 1000 & 0111 & 0001 & 1111 \\ \hline
    $B = 10$ & 8    & 7    & 1    & 15   \\ \hline
    $B = 16$ & 8    & 7    & 1    & F    \\ \hline
    \end{tabular}
\end{description}